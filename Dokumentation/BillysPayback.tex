%#####################################################################################################################
% Datei	: BillysPayback.tex
% Autor	: Byron Worms
%#####################################################################################################################
% Dokumenteinstellungen
\documentclass[bibliography=totoc,listof=totoc,version=first,a4paper,12pt,numbers=noendperiod,appendixprefix=true,
twoside,headsepline,footsepline,openright]{scrbook}

% Seitenr�nder festlegen
\usepackage[top=2.5cm,bottom=3.0cm,left=3.0cm,right=2.0cm]{geometry}

% Umlaute etc..
\usepackage[latin1]{inputenc}
\usepackage[T1]{fontenc}

% Bilder unterst�tzen
\usepackage[]{graphicx}

% Farben unterst�tzen
\usepackage{xcolor}
\usepackage{color}
\usepackage{colortbl}

% Bibliothek f�r das Quellenverzeichnisse
\usepackage[style=numeric,backend=biber,sorting=none]{biblatex}
\addbibresource{Literaturverzeichnis.bib}
%\renewcommand{\bibfont}{\small} --> finde ich zu klein

%defernumbers=true --> Das hat zu Problemen mit der Nummerierung gef�hrt!

% Verweise auf Title
\usepackage{titleref}

% Sprachunterst�tzung (Letzte Sprache wird als Richtlinie genutzt!)
\usepackage[english,ngerman]{babel} 
\usepackage{csquotes}

% Symbolunterst�tzung wie €
\usepackage{textcomp}

% Abk�rzungsverzeichnis unterst�tzen
\usepackage[printonlyused]{acronym}

% Anhang unterst�tzen
\usepackage[titletoc]{appendix}

% Erweiterte Fu�noten unterst�tzen
\usepackage[hang]{footmisc}
\setlength{\footnotemargin}{1em}

% Unterst�tzung mehrzeiliger Kommentare
\usepackage{comment}

% Folgende Objekte kapitel�bergreifend nummieren: Tabelle, Abbildungen, Fu�noten
\usepackage{chngcntr} 
\counterwithout{footnote}{chapter}

% Erweiterte Tabellen unterst�tzen
\usepackage{booktabs}
\usepackage{tabularx}
\usepackage{tabulary}
\usepackage{longtable,tabu}
\usepackage{array}
\usepackage{multicol}

% Mathematische Operationen in {} Ausdr�cken erlauben (z.B. {\linewidth-4cm})
\usepackage{calc}

% "float" Positionierung unterst�tzen
\usepackage{scrhack}
\usepackage{float}

% Erweiterte Aufzählungen unterst�tzen
\usepackage{enumitem}

% Erweiterte Tabellenlinien unterst�tzen
\usepackage{hhline,float}

% Rotationen unterst�tzen
\usepackage{rotating}

% Erweiterte Symbole unterst�tzen
\usepackage{bbding} 
\usepackage{dingbat}
\usepackage{wasysym}

\usepackage{url}

% Querformat unterst�tzen
\usepackage{pdflscape}

% Verweise in PDF-Datein aktivieren (muss das letzte importierte Package sein!)
\usepackage[colorlinks=false]{hyperref}

% Tabellen werden nicht von Latex neupositioniert
\restylefloat{table}

% Bilder werden nicht von Latex neupositioniert
\restylefloat{figure}

% PDF Version 1.6 erzwingen
\pdfminorversion=6

% Kompression ausschalten (hyperref setzt das Level auf 9!)
\pdfcompresslevel=9

% Verschiebungen und Formattierungen des Modus "twoside" aufheben
\raggedbottom

% Informationen zu dem Dokument
\newcommand{\thetitle}{Billy's Payback: Rising of the Fruits}
\newcommand{\theauthorr}{Billy's Payback}
\author{\theauthorr}
\title{\thetitle}

% Funktionen f�r das Ausblenden/Einblenden von nummerierten Überschriften im Inhaltsverzeichnis.
% Kapitel, Sektion, Untersektion
\newcommand{\hideInContents}{\addtocontents{toc}{\protect\setcounter{tocdepth}{-1}}}
\newcommand{\showInContents}{\addtocontents{toc}{\protect\setcounter{tocdepth}{2}}}

% Fake "Tabitem" erzeugen
\newcommand{\tabitem}{~~\llap{\textbullet}~~}

% Diverse Hilfsfunktionen
\newcommand{\bold}{\normalfont \bfseries}
\newcommand{\boldtext}[1]{{\normalfont \bfseries {#1}}}

% Funktion zum L�schen der Seite. Der Befehl unterscheidet automatisch zwischen einseitig und zweiseitig!
\newcommand{\myclearpage}{\if@twoside \cleardoublepage \else \clearpage \fi }

% Entfernt die Kopfzeile von automatisch generierten Leerseiten
\makeatletter
\def\cleardoublepage{\cleardoubleemptypage}
\makeatother

%---------------------------------------------------------------------------------------------------------------------
% Einstiegspunkt
%---------------------------------------------------------------------------------------------------------------------
\begin{document}
	\myclearpage
	\begingroup
		\renewcommand*{\chapterpagestyle}{empty}
		\pagestyle{empty}
		%#####################################################################################################################
% Datei	: Deckblatt.tex
% Autor	: Byron Worms
%#####################################################################################################################
\begin{titlepage}
	\thispagestyle{empty}
	\begin{center}
		\begin{figure}[t!]
			\centering
			\includegraphics[width=0.75\linewidth]{Inhalt/Titel/acagamics.png}
			\label{fig:logo_uni}
		\end{figure}
		\vspace{0.5cm}
		\large{Fakult�t f�r Informatik}\\
		\large{Acagamics - Game Student Developer Club}\\
		\vspace*{3cm}
		\huge{Wissenschaftliches Teamprojekt}\\
		\vspace*{1cm}
		\huge{\normalfont \bfseries{Billy's Payback}}\\
		\vspace*{4.0cm}
		\Large{Byron Worms, Dominique Kasper}\\
		\Large{Raik Dankworth, Ulrike Uderhardt}\\[1em]
		\Large{23.04.2015}
	\end{center}
\end{titlepage}						% Titelseite einbinden
		\cleardoubleemptypage
		%#####################################################################################################################
% Datei	: Abstrakt.tex
% Autor	: Byron Worms
%#####################################################################################################################
\let\raggedsection\centering
\section*{Kurzfassung}
Englische Kurzfassung
\selectlanguage{english}
\section*{\abstractname}
\let\raggedsection\raggedright
Deutsche Kurzfassung
\selectlanguage{ngerman}					% Kurzfassung einbinden
		\cleardoubleemptypage		
		\pagenumbering{Roman}									% R�mische Seiternummerierung	
		\tableofcontents 										% Inhaltsverzeichnis generieren
%		\include{Inhalt/ABK}									% Abk�rzungsverzeichnis einbinden
		\listoffigures{}										% Abbildungsverzeichnis generieren
%		\listoftables{}											% Tabellenverzeichnis generieren
	\myclearpage
	\endgroup
	
	\cleardoublepage  											
	\pagenumbering{arabic}										% Arabische Seiternummerierung
	
%................................................................
% Ab hier den eigentlichen Inhalt einf�gen
	%#####################################################################################################################
% Datei	: Einleitung.tex
% Autor	: Ulrike Uderhardt
%#####################################################################################################################
\chapter{Einleitung}
\label{chap:einleitung}

Schon vor Jahrzehnten haben Computerspiele in die Kinder- und Wohnzimmer unserer Welt Einzug genommen und sind heutzutage kaum noch wegzudenken. Dabei dienen Computerspiele haupts�chlich der Unterhaltung der Spieler. Die Vielfalt an unterschiedlichen Spiele-Genre ist enorm. Dazu z�hlen auch die storybasierten Spiele, bei denen um die Handlungen des Spielers eine Geschichte aufgebaut wird, in die der Spieler eintaucht. Dabei ist es allerdings h�ufig so, dass der Spieler keine gro�e Wahl hat, was den Verlauf der Story angeht. Er muss zum Beispiel Gegenst�nde aufsammeln, Gegner besiegen und bestimmten Wegen folgen, um am Ende ans Ziel zu gelangen. Dieser vorbestimmte Verlauf f�hrt dazu, dass die Entscheidungsfreiheit des Spielers eher begrenzt ist. Denn l�sst er bestimmte Handlungen aus, geht es in der Geschichte nicht weiter.
\\
Mit dieser Arbeit soll ein Prototyp entstehen, der dem Spieler die angesprochene begrenzte Entscheidungsfreiheit nimmt und einen freien Willen suggeriert. Dabei soll untersucht werden, ob Manipulationstechniken, die in 3D erfolgreich angewendet werden, auch auf 2,5D �bertragbar sind. Der Spieler soll das Gef�hl haben, alle seine Entscheidungen eigenst�ndig getroffen zu haben und sich in jeder seiner Entscheidungen frei f�hlen. Tats�chlich aber wird vorab durch ein Analyselevel erfasst, welche Vorz�ge der Spieler hat, um so den Ausgang bestimmter Situationen vorhersagen zu k�nnen. Seine vermeintliche Handlungsfreiheit soll dann durch besagte Manipulationsmethoden erreicht werden. Teilweise stammen diese aus dem Bereich der Psychologie, teilweise aus bereits vorhandenen Computerspielen, die diese Thematik auch aufgreifen.
\\
Letztendlich soll die Frage beantwortet werden, ob es m�glich ist, den Spieler auch in 2,5D so zu manipulieren, dass er seine getroffenen Entscheidungen f�r seine eigenen h�lt. Dazu findet abschlie�end ein Interview mit den Probanden statt, bei dem ausgewertet wird, inwieweit die verwendeten Manipulationen erfolgreich waren bzw. was die Probanden veranlasst hat, sich in bestimmten Situationen f�r etwas zu entscheiden.

%
% Nachfolgende Sektionen nicht im Inhaltsverzeichnis anzeigen lassen!
\hideInContents
%
%................................................................
% Hier die Sektionen f�r die Einleitung

%................................................................
%
% Die kommenden Kapitel, Sektionen und Untersektionen wieder im Inhaltsverzeichnis anzeigen
\showInContents						% Einleitung einbinden
	%#####################################################################################################################
% Datei	: Grundlagen.tex
% Autor	: Dominique Kasper
%#####################################################################################################################

\chapter{Grundlagen}
\label{chap:grundlagen}

%................................................................
\section{Manipulation}
\label{sec:manipulation}

%Ausstehende Quellen:
%Manipulation und Selbstt�uschung, Rainer Sachse {RSachseManipulationSelbsttaeuschung}

Um eine geeignete Auswahl an Manipulationen f�r den Spielprototypen treffen zu k�nnen, ist es zun�chst einmal notwendig die theoretischen Grundlagen der Manipulation sowie verschiedene Manipulationsstrategien zu kennen und zu verstehen. Aus diesem Grund soll hier sowohl transparentes und intransparentes Handeln als auch positive bzw. negative Manipulation und deren Folgen erl�utert werden. Anschlie�end erfolgt au�erdem eine Beschreibung m�glicher Manipulationsstrategien, aus denen sp�ter Manipulationen f�r den Prototypen abgeleitet werden (siehe Abschnitt \ref{sec:konzept manipulationen}).\\
Dar�ber hinaus sind diese Kenntnisse auch f�r die Auswahl der relevanten Spielereigenschaften von Nutzen.

\subsection{Transparentes und intransparentes Handeln}
\label{subsec:transparentes und intransparentes handeln}

Wenn zwei Personen (oder im Fall dieser Arbeit der Spieler und das Spiel) miteinander interagieren, wird dies als \textit{interaktionelle Handlung} bezeichnet. Dabei verfolgt jeder der beiden Interaktionspartner mit seinem Handeln eine bestimmte Absicht, d.h. er will damit ein bestimmtes Verhalten beim jeweils Anderen erreichen. Allerdings kann er auf zwei verschiedene Art und Weisen seine Absicht in eine Handlung umsetzen\cite{RSachseManipulationSelbsttaeuschung}

\begin{enumerate}
	\item Die Absicht kann durch sein Handeln erkennbar werden.
	\item Die Absicht kann verschleiert werden.
\end{enumerate}

\subsubsection{Transparentes Handeln}

Wird die Absicht durch das Handeln einer Person erkennbar, wird dies als \textit{transparentes Handeln} bezeichnet. Bei dieser Art des Handelns, kann der Gegen�ber selbst entscheiden, ob er der Absicht der Person entspricht oder nicht. Er hat somit echte Entscheidungsfreiheit, da er sich auf die wahre Absicht der Handlung einstellen kann. Ein Beispiel hierf�r w�re "Ich m�chte, dass du dich um mich k�mmerst.". Aussagen wie "Wenn sich doch jetzt nur jemand um mich k�mmern w�rde ...", wo der Interaktionspartner also zwischen den Zeilen lesen muss, werden zwar auch noch zu transparentem Handeln gez�hlt, k�nnen jedoch nicht mehr als v�llig transparent betrachtet werden, da hier ein zus�tzlicher Aufwand notwendig ist, um die Absicht zu entschl�sseln.\cite{RSachseManipulationSelbsttaeuschung}

\subsubsection{Intransparentes Handeln}

Wird die wahre Absicht jedoch verschleiert, handelt es sich um \textit{intransparentes Handeln}. In diesem Fall ist es sogar notwendig dem Gegen�ber eine glaubhafte, falsche Absicht f�r das eigene Handeln zu liefern und zwar eine, die er mit hoher Wahrscheinlichkeit auch akzeptiert. Welche Absicht das ist, ist wiederum von der Pers�nlichkeit des Interaktionspartners abh�ngig. Ist er beispielsweise jemand, der dazu neigt schnell Mitleid zu empfinden, ist es sinnvoll das Handeln mit einem schlechten Befinden zu begr�nden. So w�re es eine M�glichkeit zu sagen "Mir geht es heute wirklich schlecht, k�mmere dich bitte um mich.". Die Abh�ngigkeit des Erfolges einer Manipulation von den Eigenschaften des Gegen�bers wird in Abschnitt \ref{sec:konzept spielereigenschaften} im Bezug auf die Spielereigenschaften noch von Bedeutung sein. Durch dieses Hinwegt�uschen �ber die wahren Gr�nde des Handeln wird der Interaktionspartner jedoch auch seiner Entscheidungsfreiheit beraubt, denn da er nun die eigentliche Absicht nicht kennt, hat er nicht die M�glichkeit sich daf�r oder dagegen zu entscheiden. Dar�ber hinaus sind gut erdachte, falsche Absichten meist sehr zwingend, womit er sich verpflichtet f�hlt, diesen auch nachzugeben.\cite{RSachseManipulationSelbsttaeuschung}

\subsection{Definition der Manipulation}
\label{subsec:definition manipulation}

Auf der Grundlage des zuvor behandelten transparenten und intransparenten Handelns, l�sst sich nun gut der Begriff der Manipulation definieren. So hat diese laut Rainer Sachse drei wesentliche Eigenschaften\cite{RSachseManipulationSelbsttaeuschung}:

\begin{itemize}
	\item Intransparentes Handeln
	\item Veranlassung des Interaktionspartners zu einer Handlung, die er ohne Manipulation nicht ausf�hren w�rde
	\item Einschr�nkung der Entscheidungsm�glichkeiten des Interaktionspartners durch die T�uschung hinsichtlich der wahren Handlungsgr�nde
\end{itemize}

Dar�ber hinaus nimmt er eine Unterscheidung der Manipulationen hinsichtlich der Folgen des durch die Manipulation erfolgten Handelns f�r die manipulierte Person vor. Diese sollen im Folgenden erl�utert werden.

\subsection{Positive Manipulation}
\label{subsec:positive manipulation}

Die Voraussetzungen f�r eine positive Manipulation sind immer dann gegeben, wenn der Interaktionspartner zwar dazu veranlasst wird etwas zu tun, das er eigentlich nicht tun wollte, dieses Handeln jedoch auch einen positiven Effekt f�r ihn hat, d.h. auch einige seiner Ziele und Motive befriedigt werden. Ein Beispiel hierf�r w�re, wenn eine Person mit dem Ziel auf eine Feier geht, m�glichst viele Leute zu treffen und sich mit ihnen zu unterhalten. Dort angekommen zieht jedoch nur eine einzige Person ihre gesamte Aufmerksamkeit auf sich, indem sie laut und aufdringlich eine sehr spannende und unterhaltsame Geschichte erz�hlt. So kommt die erste Person zwar nicht dazu sich mit vielen Leuten zu unterhalten, aber eines ihrer Motive wird dabei dennoch erf�llt: sie wird gut unterhalten. Die Manipulation erfolgt also �ber das Motivsystem des Interaktionspartners.\cite{RSachseManipulationSelbsttaeuschung}

\subsubsection{Vorteile}

Positive Manipulationsstrategien l�sen keine unmittelbaren negativen Emotionen aus. So ist der Interaktionspartner meist erst ver�rgert, wenn er die Manipulation erkennt oder sich ausgenutzt f�hlt. Aus diesem Grund sind solche Strategien nur wenig \textit{interaktionstoxisch}, d.h. sie sind nur in geringem Ma�e beziehungssch�digend.

\subsubsection{Nachteile}

Auf der anderen Seite unterliegt der Interaktionspartner bei einer positiven Manipulation nur einem geringen Handlungszwang, da ein gegenteiliges Handeln kaum negative Folgen f�r sie hat.

\subsection{Negative Manipulation}
\label{subsec:negative manipulation}

Eine negative Manipulation findet immer dann statt, wenn der Interaktionspartner zu einer Handlung veranlasst wird, die seinen eigenen Zielen und Motiven zuwiderl�uft, ohne dass er etwas Positives zur�ckbekommt. Aus diesem Grund f�hrt diese Art von Manipulation oft zu Frustration und �rger seitens der manipulierten Person. Als Beispiel soll hier erneut die Person dienen, die auf eine Feier geht, um m�glichst viele Leute kennen zu lernen sowie mit ihnen unterhaltsame Gespr�che �ber Themen zu f�hren, die sie interessieren. Nun wird sie jedoch den gesamten Abend von einer einzigen Person, die vorgibt, dass es ihr schlecht geht und sie F�rsorge ben�tigt, festgehalten und ausschlie�lich mit uninteressanten Gespr�chsthemen gelangweilt. Auf diese Art und Weise wird die erste Person von ihrem urspr�nglichen Vorhaben abgehalten und ihr Bed�rfnis gut unterhalten zu werden, wird ebenfalls nicht erf�llt. Die Manipulation erfolgt also �ber das Normsystem des Interaktionspartners, wodurch dieser sich zu einer Handlung verpflichtet f�hlt, da er denkt, dass ein gegenteiliges Handeln nicht zu rechtfertigen w�re.\cite{RSachseManipulationSelbsttaeuschung}

\subsubsection{Vorteile}

Bei negativen Manipulationen herrscht meist ein hoher Handlungszwang f�r den Interaktionspartner. Dieser ist umso h�her, desto gr��er das schlechte Gewissen bei einem Zuwiderhandeln w�re.

\subsubsection{Nachteile}

Negative Manipulationen sind jedoch auch stark \textit{interaktionstoxisch}, da sich der Interaktionspartner aufgrund des gro�en Handlungsdrucks schnell ausgenutzt und unzufrieden f�hlt.\\
Beispiele f�r solch negative Manipulationsstrategien sind:\cite{RSachseManipulationSelbsttaeuschung}

\begin{itemize}
	\item Produzieren von Symptomen
	\item Kritik
	\item Druck aus�ben
	\item Drohungen
	\item (St�ndige) N�rgeleien
	\item Nachtragendes Verhalten
\end{itemize}

\subsection{Dosierung von Manipulationen}
\label{subsec:dosierung von manipulationen}

In seinem Buch \textit{Manipulation und Selbstt�uschung}\cite{RSachseManipulationSelbsttaeuschung} spricht Rainer Sachse auch von der gro�en Bedeutung der Dosierung von Manipulationen. So wirken sie sich bei authentischem Verhalten und gem��igtem Einsatz sogar meist g�nstig aus. Hierbei ist die sogenannte \textit{Reziprozit�tsregel} zu beachten, welche besagt, dass zwei Personen nur dann eine gute und stabile Beziehung f�hren k�nnen, wenn beide ungef�hr in gleichem Ma�e oder entsprechend ihrer jeweiligen Erwartungen von dieser profitieren. Aus diesem Grund sind Manipulationen solange unproblematisch wie jeder auf seine Kosten kommt und die jeweiligen Erwartungen und Bed�rfnisse nicht zu sehr missachtet werden. Entsteht jedoch ein Ungleichgewicht, bei dem der eine Partner mehr als der andere manipuliert und seinen Willen durchsetzt, f�hrt dies �ber kurz oder lang zur Unzufriedenheit und Beziehungsproblemen. Diese Auswirkungen werden als \textit{interaktionelle Kosten} bezeichnet. Wie schnell diese auftreten h�ngt erneut von der Pers�nlichkeit der Zielperson ab, da einige langsamer und einige schneller negativ reagieren. In jedem Fall ist das Endergebnis jedoch, dass sich der unzufriedene Partner nicht mehr manipulieren l�sst oder sogar genau das Gegenteil tut.

\subsection{Einfluss pers�nlicher Faktoren auf den Manipulationserfolg}
\label{subsec: einfluss pers�nlichkeit auf manipulationserfolg}

Wie bereits in den vorherigen Abschnitten erw�hnt, ist der Erfolg einer Manipulation nicht zuletzt auch von der Zielperson abh�ngig. So gibt es im Allgemeinen sowohl Faktoren, die ihre Anf�lligkeit gegen�ber Manipulationen erh�hen als auch solche, die ihre Anf�lligkeit verringern. Im Folgenden sollen diese erl�utert werden.\cite{RSachseManipulationSelbsttaeuschung}

\subsubsection{Anf�lligkeit reduzierende Faktoren}

Interaktionspartner sind weniger anf�llig gegen�ber Manipulationen, wenn:

\begin{itemize}
	\item Sie die wahren Absichten anderer gut erkennen k�nnen
	\item Sie unnat�rliches Verhalten anderer intuitiv erkennen und es schnell zu \textit{St�rgef�hlen} kommt
	\item Sie einen starken Wunsch nach Autonomie haben
	\item Sie sehr selbstbewusst sind und sich nicht so schnell beeindrucken lassen
	\item Sie zwar ethischen Prinzipien folgen, aber der Ansicht sind, dass jeder f�r sich selbst die Verantwortung tr�gt
	\item Sie nicht den Zw�ngen eines \textit{Helfer-Syndroms} unterliegen, d.h. sie andere stets vor Unheil bewahren bzw. retten wollen
\end{itemize}

Mit St�rgef�hlen ist hier gemeint, wenn die Person das Gef�hl hat nicht auf ihre Kosten zu kommen bzw. ihre Ziele nicht zu erreichen und/oder etwas zu tun, das sie eigentlich gar nicht tun m�chte. Dar�ber hinaus bemerken autonome Interaktionspartner meist schneller, wenn sie manipuliert werden und reagieren entsprechenden mit Ablehnung.

\subsubsection{Anf�lligkeit verst�rkende Faktoren}

Interaktionspartner sind anf�lliger gegen�ber Manipulationen, wenn:

\begin{itemize}
	\item Sie die wahren Absichten anderer nicht gut erkennen k�nnen	
	\item Es ihnen schwer f�llt unnat�rliches Verhalten zu erkennen
	\item Es bei ihnen nicht zu St�rgef�hlen kommt oder sie diese nicht wahrnehmen
	\item Sie den Zw�ngen eines Helfer-Syndroms unterliegen
	\item Sie dazu neigen anderen stets alle W�nsche von den Augen ablesen und erf�llen zu wollen
	\item Sie Konflikte mit anderen scheuen und es ihnen schwer f�llt sich f�r die eigenen W�nsche einzusetzen
	\item Sie sich streng nach Normen richten, die besagen, dass sie anderen stets helfen und nicht egoistisch sein sollen
	\item Sie sich schnell verpflichtet f�hlen (z.B. wenn ihnen ein Gefallen getan wurde)
\end{itemize}

\subsection{Interaktionsspiele}
\label{subsec:interaktionsspiele}

In \textit{Manipulation und Selbstt�uschung} von Rainer Sachse\cite{RSachseManipulationSelbsttaeuschung} werden ebenfalls sogenannte \textit{Interaktionsspiele} ausf�hrlich abgehandelt. So werden diese von ihm unter anderem als komplexe Manipulationsstrategien bezeichnet, die eingesetzt werden, um bestimmte Interaktionsziele zu erreichen. Sie werden von Personen erlernt und dann lebenslang ge�bt und verfeinert, so dass diese sie mitunter sogar ganz automatisch ausf�hren, ohne dass sie viel dar�ber nachdenken m�ssen. Dar�ber hinaus werden sie von der entsprechenden Person meist nicht mehr angezweifelt, da sie sich bereits mehrfach gut bew�hrt haben. Im Folgenden sollen die Interaktionsspiele au�erdem nach Sachse eingeteilt werden.

\subsubsection{Attraktivit�tsspiele}

Bei Attraktivit�tsspielen m�chte die manipulierende Person sich in ein besonders gutes Licht r�cken. Meist sollen auf diese Art und Weise Anerkennung, Aufmerksamkeit und Bewunderung durch andere erreicht werden. Es sind jedoch auch beliebige andere Zwecke m�glich. Des Weiteren k�nnen solche Spiele in verschiedene Unterarten unterteilt werden:

\begin{itemize}
	\item Darstellung der Person als jemanden mit vielen positiven und m�glichst keinen negativen Eigenschaften, die auch durch Abwertung anderer unterst�tzt werden kann (\textit{Mords-Molly-Spiel})
	\item Sehr aufwendige Demonstration der Attraktivit�t einer Person, z.B. durch teure Kleidung und Make-Up (\textit{Attraktivit�t-Spiel})
	\item Besondere Betonung sexueller Reize, z.B. durch wenig Kleidung oder laszives Verhalten (\textit{Sexy-sein-Spiel})
	\item Pr�sentation der eigenen Person und Geschichten als etwas besonders spannendes und unterhaltsames, um anderen im Ged�chtnis zu bleiben (\textit{Unterhaltsam-sein-Spiel})
\end{itemize}

\subsubsection{Armes-Schwein-Spiele}

Bei dieser Art von Interaktionsspiel m�chte die betreffende Person als besonders hilfebed�rftig, schwach und leidend wahrgenommen werden, um m�glichst viel Hilfe, F�rsorge und Mitleid von anderen entgegengebracht zu bekommen. Auch hier ist eine weitere Unterteilung m�glich:

\begin{itemize}
	\item Darstellung der Person als besonders stark belastet und hilflos, so dass sich andere verpflichtet (oder herausgefordert) f�hlen ihr zu helfen (einfaches \textit{Armes-Schwein-Spiel})
	\item Erweiterung des Armes-Schwein-Spiels um den Aspekt, dass die Person als jemand bewundert werden soll, der sein starkes Leiden bzw. seine Belastung erfolgreich gemeistert hat (\textit{Heroisches-armes-Schwein-Spiel})
\end{itemize}

Auf Arme-Schweine-Spiele sprechen insbesondere Menschen gut an, die ein \glqq Helfer-Syndrom\grqq{} oder schnell Mitleid haben sowie solche sich durch unl�sbare Aufgaben anderer herausgefordert f�hlen. Solche Spiele sind dar�ber hinaus allgemein hoch manipulativ, wodurch sie in Ma�en eingesetzt werden sollten, um den Interaktionspartner nicht in die Flucht zu jagen.

\subsubsection{Opfer-Spiele}

Bei Opfer-Spielen geht es f�r die ausf�hrende Person in erster Linie darum jegliche Schuld f�r eigenes Versagen oder Nicht-Handeln von sich zu weisen und die Verantwortung an den Interaktionspartner zu �bertragen. Es werden folgende Unterarten unterschieden:

\begin{itemize}
	\item Darstellung der Person als Opfer ungl�cklicher Zuf�lle bzw. des Schicksals oder von bestimmten Personen (\textit{\glqq Opfer der Umst�nde oder anderer Personen\grqq{} - Spiel})
	\item Die Person gibt vor, dass sie etwas verl�sslich tun wird, tut es dann jedoch nicht und schiebt die Schuld daf�r auf �u�ere Umst�nde oder Personen (\textit{Sabotage-Spiel})
	\item Darstellung der Person als extrem von �u�eren Umst�nden oder Personen beeintr�chtigt bzw. ungerecht behandelt, aber trotzdem erfolgreich im Erreichen ihrer Ziele (\textit{M�rtyrer-Spiel})
	\item Darstellung der Person als au�ergew�hnlich oft von schlechten, durch �u�ere Umst�nde oder Personen verursachte Ereignisse betroffen (\textit{\glqq Immer ich\grqq{} - Spiel})
	\item Beteiligung einer Person an einem gegenseitigen Mobbing, wobei anschlie�end die gesamte Schuld auf den Interaktionspartner geschoben wird (\textit{Mobbing-Spiel})
\end{itemize}

\subsubsection{Regel-Setzer-Spiele}

Bei diesen Interaktionsspielen versucht die manipulierende Person dem Interaktionspartner vorzuschreiben, was er zu tun und/oder zu lassen hat. F�r sie ist dabei allein der Grund ausreichend, dass sie es so m�chte und es eben so getan bzw. nicht getan werden muss. Unterteilt wird hierbei in:

\begin{itemize}
	\item Eine Person f�hlt sich dazu autorisiert Regeln f�r andere aufzustellen und jedes Nicht-Einhalten auf irgendeine Weise zu ahnden (einfaches \textit{Regel-Setzer-Spiel})
	\item Eine Person f�hlt sich dazu berufen Interaktionspartner zu leiten, auf den rechten Weg zu f�hren und daf�r auch stark zu ma�regeln oder zu diskriminieren, sollten sie sich nicht ihren Ansichten anpassen (\textit{Moses-Spiel})
	\item Darstellung der Person als sehr hilfsbed�rftig und Erl�sung durch sehr gro�e Anstrengung anderer w�nschend, wobei nur der m�glichst unerreichbare Aufwand der Rettung z�hlt und nicht diese an sich (\textit{Dornr�schen-Spiel})
	\item Vermeiden des Kontaktes mit dem Interaktionspartner durch Nicht-Reagieren, z.B. nichts sagen, nicht angucken (\textit{\glqq Distanz halten\grqq{} - Spiel})
\end{itemize}

Die meisten Regel-Setzer-Spiele sind hochgradig interaktionstoxisch und werden vor allem von autonomen Pers�nlichkeiten boykottiert.

\subsubsection{Bl�d-Spiele}

Bei Bl�d-Spielen geht es darum, dass sich die manipulierende Person l�stiger Aufgaben entledigen will, indem sie vorgibt f�r diese \glqq zu bl�d\grqq{} zu sein. Es wird dabei zwischen folgenden Unterarten unterschieden:

\begin{itemize}
	\item Die Person stellt sich als unf�hig dar, die betreffende Aufgabe zu erledigen ohne Schaden anzurichten und versucht den Interaktionspartner zu dieser zu �berreden, indem er ihm schmeichelt wie gut er diese erf�llen k�nne (einfaches \textit{Bl�d-Spiel})
	\item Variante des einfach Bl�d-Spiels, bei dem nicht Aufgaben, sondern Entscheidungen an den Interaktionspartner delegiert werden (\textit{\glqq Entscheidung abgeben\grqq{} - Spiel})
\end{itemize}

Das Abgeben von Entscheidungen funktioniert hier besonders gut bei machtorientierten Interaktionspartnern oder Personen mit einem Helfer-Syndrom. Dar�ber hinaus kann diese Strategie auch der Bindung an einen Partner zutr�glich sein, da dieser auf diese Art und Weise viel Einfluss bekommt und keine Konflikte mit ihm entstehen, da ihm nicht widersprochen wird. Dies funktioniert jedoch nicht bei allen Menschen, da nicht jeder unselbstst�ndige Pers�nlichkeiten vorzieht.

\section{Zusammenfassung}
\label{sec:zusammenfassung grundlagen}

Im aktuellen Kapitel erfolgte zun�chst die Definition einer Manipulation �ber das intransparente Handeln und die damit verbundene Einschr�nkung der Entscheidungsm�glichkeiten des Interaktionspartners sowie die Veranlassung des selbigen zu einer Handlung, die er ohne die Manipulation nicht ausf�hren w�rde. Anschlie�end werden sowohl die Vor- und Nachteile negativer und positiver Manipulationen erl�utert als auch der Einfluss der Manipulationsdosierung und personenbezogener Faktoren beschrieben.\\
Somit lassen sich zusammenfassend einige allgemeine Voraussetzungen f�r eine erfolgreiche Manipulation ableiten. So ist es notwendig sowohl �ber positive als auch negative Manipulationsstrategien zu verf�gen. Dabei sollten zwar haupts�chlich Erstere zum Einsatz kommen, aber im Falle ihres Versagens, besteht die M�glichkeit auf Letztere zur�ckzugreifen, um m�glicherweise doch noch einen Erfolg zu erzielen. Auch ist darauf zu achten, dass Manipulationen nur in gem��igter Dosierung angewandt werden. Dar�ber hinaus ist es stets von Vorteil �ber die Eigenschaften bzw. Pers�nlichkeit der Zielperson Bescheid zu wissen, da unterschiedliche Charakteristika hier auch zu unterschiedlicher Anf�lligkeit bzw. Immunit�t gegen�ber bestimmten Manipulationen f�hren.\\
Den Abschluss bilden die von Rainer Sachse angef�hrten Interaktionsspiele.

%................................................................						% Grundlagen einbinden
	%#####################################################################################################################
% Datei	: VerwandteArbeiten.tex
% Autor	: Dominique Kasper
%#####################################################################################################################
\chapter{Verwandte Arbeiten}
\label{chap:verwandte arbeiten}

In 3D-Spielen sind Manipulationen des Spielers hinsichtlich seiner Entscheidungen bereits g�ngige Praxis. Dabei werden sie in einigen Spielen gut und in anderen weniger gut umgesetzt. So sollen in der aktuellen Arbeit die guten Umsetzungen als Vorbild und die weniger guten als Ansatz f�r Verbesserungen dienen. \\
Dar�ber hinaus existieren auch wissenschaftliche Arbeiten und Artikel zum Thema der Spielermanipulation, anhand derer sowohl sinnvolle Manipulationen ausgew�hlt als auch die unterschiedlichen Spielereigenschaften ermittelt werden sollen.

%................................................................
% Hier die Sektionen f�r verwandte Arbeiten

\section{Player Manipulation}
\label{sec:player manipulation paper}

In dem Paper \textit{Player Manipulation}\cite{SiewNareyekPlayerManipulation} setzen sich Zi Xu Siew and Alexander Nareyek mit dem Problem auseinander, dass Spieleentwickler immer nur eine begrenzte Menge an relevanten und interessanten Spielinhalten erstellen k�nnen und den Spieler somit gezielt zu diesen f�hren m�ssen, ohne jedoch die Interaktivit�t des Spiels zu vernachl�ssigen. Um diese Problematik zu l�sen, schlagen sie eine Reihe von Manipulationen vor:

\begin{itemize}[itemsep=0em]
	\item Verpflichtungen bzw. Verpflichtungsgef�hle
	\item Revanchieren f�r eine Gef�lligkeit
	\item Zeitdruck
	\item Umgebungshinweise
	\item Gruppenzwang
	\item Beeinflussung durch eine Autorit�t
	\item Affektive Hintergrundmusik
	\item Anfragen von nahestehenden Charakteren (Partnern)
\end{itemize}
\noindent
Diese werden au�erdem in einem entsprechenden Testrahmen evaluiert, wobei sich die Anfragen durch nahestehende Partner und die Beeinflussung durch Autorit�ten als am erfolgreichsten sowie Umgebungshinweise, affektive Hintergrundmusik und der Gruppenzwang als am wenigsten erfolgreich erwiesen. Es wird jedoch darauf hingewiesen, dass die Ursachen hierf�r auch bei den kleinen Versuchsgr��en und wenigen Kontrollexperimenten sowie der geringen Erfahrung in diesem Forschungsgebiet liegen k�nnten.

\section{Die Psycho-Tricks der Spiele-Designer: Das Spiel in deinem Kopf}
\label{sec:die psychotricks der spieledesigner}

Bei \textit{Die Psycho-Tricks der Spiele-Designer: Das Spiel in deinem Kopf}\cite{PsychoTricksPCGames} handelt es sich um einen Artikel auf der Webseite der Zeitschrift PCGames, welcher sich mit der Beeinflussung des Unterbewusstseins des Spielers besch�ftigt. Hierbei werden folgenden Manipulationen als sehr erfolgsversprechend beschrieben.

\subsection*{Mehr Motivation f�r Quests durch Anfangserfolge}

Bei dieser Methode wird dem Spieler bei dem Beginn einer Aufgabe vorgegaukelt, dass er bereits einen Teilerfolg errungen hat. Dies geschieht beispielsweise bei Sammelaufgaben dadurch, dass er schon kurz vor Vergabe der Quest einen der gesuchten Gegenst�nde findet und somit das Gef�hl vermittelt bekommt sich bereits mitten in der Aufgabe zu befinden. Auf  diese Art und Weise wird die Motivation des Spielers, diese auch zu Ende zu bringen, gesteigert.

\subsection*{Questtageb�cher}

Questtageb�cher sollen daf�r sorgen, dass der Spieler durch die st�ndige Erinnerung den Drang versp�rt die entsprechenden Aufgaben endlich abzuschlie�en, was auch als \textit{Cliffhanger- oder Zeigarnik Effekt} bekannt ist. Ebenfalls n�tzlich ist hier der \textit{Ovsiankina-Effekt}, der das qu�lende Gef�hl beschreibt, das die meisten Menschen bekommen, wenn sie eine Aufgabe annehmen und dann abbrechen.

\subsection*{Farbcodierung}

Wenn im Spiel immer wieder bestimmte Objekte oder Ereignisse mit entsprechenden Farben markiert werden, so dass der Spieler nach einer Weile die Farbe automatisch mit dem jeweiligen Objekt oder Ereignis assoziiert, wird dies Farbcodierung oder auch Color-Coding genannt. Gute Beispiele hierf�r sind rote F�sser, bei denen somit suggeriert wird, dass diese zu zerst�ren und ggf. explodieren bzw. gef�hrlich sind. Immer in einem dezenten Blau gekennzeichnete Ausg�nge einzelner Levels sind hier ebenfalls aufzuf�hren. Nicht au�er Acht zu lassen ist auch der rein psychologische Effekt einiger Farben auf den Menschen. So wirkt Rot z.B. provozierend und l�st Angriffsreflexe aus.

\subsection*{Gerechte-Welt-Glauben}

Die Bezeichnung \textit{Gerechte-Welt-Glauben} beschreibt einen Effekt, der eintritt, wenn ein bestimmter Charakter im Spiel durch seine Taten als schlecht, unmoralisch oder anderweitig ungerecht dargestellt wird und es so dem Spieler leichter f�llt ihn auf eine furchtbare Art und Weise zu bestrafen. Er kann somit dazu eingesetzt werden, um den Spieler dazu zu treiben einen in der weiteren Geschichte unerw�nschten Charakter zu beseitigen.

\subsection*{Theorie des sozialen Vergleiches}

In dieser Theorie begr�ndet sich die M�glichkeit in vielen Spielen die Ergebnisse der Spieler in sozialen Netzwerken zu ver�ffentlichen, denn sie besagt, dass Menschen stets den Drang haben sich mit anderen zu messen. Auf diese Art und Weise soll das Posten eigener Ergebnisse andere dazu animieren diese zu �bertreffen und anschlie�end ebenfalls �ffentlich zu machen. Dies f�hrt schlie�lich zu einer Endlosschleife des gegenseitigen �bertreffen-Wollens. So wird ein Anreiz f�r das Bew�ltigen von Herausforderungen geschaffen.

\subsection*{Anreiz durch die Beute}

Hierbei steckt der Anreiz nicht etwa in der Beute selbst, sondern vielmehr in der Erwartung und Spekulation dar�ber, was der Spieler als n�chstes finden k�nnte und genau dies motiviert ihn dann auch dazu nach weiteren Gegenst�nden Ausschau zu halten. Au�erdem lassen sich beispielsweise immer zum Zeitpunkt des Findens der Beute ert�nende Soundeffekte mit dem beim Sammeln empfundenen Gl�cksgef�hl verkn�pfen, wodurch dieses allein durch das Abspielen dieser T�ne hervorgerufen werden kann. Dies gilt auch f�r andere Reize.

\subsection*{Glaubw�rdigkeit der Charaktere}

Um auf durch Emotionen gesteuerte Manipulationen zur�ckgreifen zu k�nnen, bedarf es Charaktere, die solche im Spieler entstehen lassen und mit denen er daher eine parasoziale Beziehung eingehen kann. Hierzu werden im Artikel Erkenntnisse aus der Kommunikationspsychologie genannt, die besagen, dass Menschen eher attraktiven Personen ihr Vertrauen schenken und Nichtspielercharaktere menschlicher wirken, wenn sie das Wort direkt an den Spieler richten und ihm dabei in die Augen sehen.\\

Dar�ber hinaus wird die der Immersion sehr zutr�gliche Wirkung von Effekten wie beispielsweise Dreck auf der virtuellen Linse erl�utert. Auch die Wirkung kaum h�rbarer Hintergrundger�usche bzw. -musik auf die Stimmung des Spielers finden Erw�hnung.

\section{The Stanley Parable}
\label{sec:the stanley parable}

Bei \textit{The Stanley Parable}\cite{SpielStanleyParable} handelt es sich um ein Adventure, welches sich selbst nicht so ernst nimmt. Dabei geht es um den B�romitarbeiter Stanley, der jeden Tag seiner eint�nigen Arbeit nachgeht, bei der er stets Anweisungen �ber einen Monitor bekommt, die ihm mitteilen, welche Tasten er auf seiner Tastatur zu dr�cken hat. Doch eines Tages bleibt der Bildschirm leer und auch all seine Kollegen sind verschwunden, wodurch er den Rest der B�ror�ume erkunden muss, um herauszufinden, was passiert ist. Erz�hlt wird sein Voranschreiten au�erdem von einem Erz�hler aus dem Off. Dieser gibt dem Spieler somit eigentlich konkrete Anweisungen, wie er sich zu verhalten hat. Allerdings ist es auch m�glich diese zu verweigern und einen anderen Weg einzuschlagen, wodurch der Erz�hler gezwungen wird seine Geschichte anders zu erz�hlen. Dies f�hrt jedoch irgendwann zwangsl�ufig zu Verwirrungen im Verlauf der Geschichte, bei denen scheinbar selbst der Erz�hler nicht mehr zu wissen scheint, wo es langgeht. Auf diese Art und Weise kann der Spieler 18 verschiedene Enden erspielen. \\
W�hrend des Spiels kommen sowohl unscheinbare als auch eher offensichtliche Mittel zum Einsatz, um den Spieler glauben zu lassen, dass er alleine bestimmt, obwohl eigentlich der gr��te Teil vom Spiel festgelegt wird und der Spieler nur an wenigen Stellen selbst den Verlauf der Geschichte �ndern kann. Zu den unterschwelligen Manipulationen geh�ren z.B. eine gro�e Auswahl an T�ren, von denen alle an dieselbe Stelle f�hren oder (rote) Lichter, die die Aufmerksamkeit des Spielers auf einen anderen Weg leiten. Auch wird hier damit gearbeitet, dass der Spieler sich dann und wann selbst die Schuld daf�r gibt, dass seine eigenst�ndigen Entscheidungen zu einem Chaos in der Geschichte gef�hrt haben und er somit das vorgesehene Geschehen nicht sehen kann. Eher klar erkennbar sind dahingegen die vielen verschlossenen T�ren, die dem Spieler kaum eine Wahl lassen oder die Linie, die in einer der Varianten den Spieler zum Ende des Spiels f�hrt.
%
%\section{The Vanishing of Ethan Carter}
%\label{sec:the vanishing of ethan carter}
%
%In \textit{The Vanishing of Ethan Carter} erh�lt der Detektiv Paul Propsero einen verst�renden Brief des Jungen Ethan Carter aus Red Creek Valley. Dieser scheint in Gefahr zu sein, weshalb Paul sich aufmacht, um herauszufinden, was geschehen ist. So besteht das Spiel gr��tenteils daraus nach Hinweisen zu suchen und ihnen nachzugehen, um die Puzzleteile des R�tsels um Ethan Carter zusammenzusetzen. Hierbei setzen die Entwickler des Spiels haupts�chlich auf visuelle Umgebungshinweise, um die Aufmerksamkeit des Spielers auf bestimmte Teile der Spielwelt zu lenken. Zu diesen geh�ren beispielsweise schimmernde Partikel, die �berall dort vom Boden aufsteigen, wo sich bestimmte Szenenausschnitte befinden, die richtig nummeriert und somit zu einer passenden Szene zusammengef�gt werden m�ssen. \cite{VideoEthanCarter} Dennoch besitzt der Spieler hier tats�chlich die Freiheit die Welt frei und in beliebiger Reihenfolge zu erkunden, um zu einer L�sung zu gelangen. Allerdings zeigen sich die Grenzen dieser Freiheit darin, dass trotzdem keine R�tsel, seien sie aufgrund der restlichen Erz�hlung auch noch so unbedeutend, ausgelassen werden d�rfen, m�chte der Spieler zum Ende des Spiels gelangen. Auf diese Art und Weise wird ihm schlie�lich recht deutlich bewusst, dass er nur der Illusion erlegen ist, er habe die freie Wahl, was er in The Vanishing of Ethan Carter tut und was nicht.\cite{TestEthanCarterGameStar}

\section{Fable}
\label{sec:fable}

In \textit{Fable} erlebt der Spieler die Geschichte des Hauptcharakters vom kleinen Jungen zum gro�en Krieger. Dabei kann er sich entscheiden, ob er sich auf die Seite des Guten oder des B�sen schl�gt. Im Gro�en und Ganzen wirken sich hier die Entscheidungen des Spielers tats�chlich auf bestimmte Ebenen des Spiels aus, da sich entsprechend der guten oder b�sen Ausrichtung der Charakter und auch die Quests �ndern. Allerdings findet hier zumindest eine versuchte Manipulation des Spielers statt, da die  Mitmenschen des Charakters negativ auf eine unmoralische Entwicklung bzw. positiv auf eine moralische Entwicklung reagieren und er anfangs auch von seinem Vater f�r schlechtes Verhalten ger�gt wird. Hier ist allerdings zu beachten, dass derlei Wertungen nicht auf jeden Spielertyp Auswirkungen haben. Auch wird das Entscheidungsprinzip an einigen Stellen korrumpiert, da es manchmal n�tig ist sich auf eine ganz bestimmte Weise zu verhalten, um im Spiel voranschreiten zu k�nnen, wodurch eine vollst�ndige Freiheit im Verlauf der Geschichte letztendlich doch nicht l�ckenlos gegeben ist.\cite{KZieroldComputerspielanalyse}
%
%\section{Die Versuchung}
%\label{sec:die versuchung}
%
%Bei \textit{Die Versuchung} handelt es sich um einen interaktiven Spielfilm, in dem der Spieler dem Psychologen Doktor Turner dabei helfen soll die Hintergr�nde der seltsamen Ereignisse zwischen seiner tragischen und sch�nen Patientin Allison, ihrem Ehemann und ihrer gut aussehenden Psychologin zu entwirren. Letztere wurde von Allisons Ehemann angeheuert nachdem ihre einzige Tochter Jody bei einem schlimmen Autounfall ums Leben kam. W�hrend der Spieler die Detektivarbeit �bernimmt, wird au�erdem seine Pers�nlichkeit von Doktor Turner anhand von Fragen zu Personen und Handlungen des Spiels sowie Bildern analysiert. Letzteres wird auch als TAT (Thematischer Apperzeption Test) bezeichnet, mit dem sich die Wahrnehmung der Umwelt und das pers�nlichkeitsbedingte Eigenerleben einer Person bestimmen lassen. Abh�ngig von den Ergebnissen der Befragungen ver�ndert sich auch der Verlauf der Geschichte, so dass sieben verschiedene Enden erreicht werden k�nnen. Dar�ber hinaus wird dem Spieler das Gesamtergebnis der psychologischen Analyse am Ende des Spiels mitgeteilt.\cite{DieVersuchungAdventureTreffKritik, DieVersuchungAftermathmedia}

\section{Zusammenfassung}
\label{sec:zusammenfassung verwandte arbeiten}

Im Allgemeinen l�sst sich sagen, dass sowohl eine kleinere Auswahl an wissenschaftlichen Arbeiten bzw. Artikeln zum Thema Spielermanipulation als auch einige Spiele, in denen solche Manipulationen bereits eingesetzt werden, existieren.\\
So grenzen die Autoren in \textit{Player Manipulation} acht Grundkategorien von Manipulationen ein, die sie innerhalb eines kleineren Testrahmens auch hinsichtlich ihres Erfolges evaluieren. Aufbauend auf ihren Erkenntnissen sollen in der aktuellen Arbeit f�r einen 2,5D-Kontext geeignete Manipulationen ausgew�hlt und innerhalb eines zusammenh�ngenden, in sich geschlossenen Spielprototypen weiter getestet werden. Als zus�tzliche Bereicherung bei der Auswahl und Einsch�tzung dient der Online-Artikel \textit{Die Psycho-Tricks der Spiele-Designer: Das Spiel in deinem Kopf} der PCGames, welcher auf weitere Manipulationsans�tze, wie die Theorie des Gerechte-Welt-Glaubens und des sozialen Vergleiches, aufmerksam macht.\\
Dar�ber hinaus wurden ein paar Beispiele f�r Spiele, in denen Spielermanipulationen zum Einsatz kommen, zur Orientierung f�r diese Arbeit herangezogen. So wird in \textit{The Stanley Parable} h�ufig dem Spieler die Schuld daf�r eingeredet, dass der Spielverlauf im Chaos endete, um ihn beim n�chsten Mal auf den vermeintlich richtigen Weg zu f�hren, w�hrend in \textit{Fable} versucht wurde den Spieler auf der moralischen Ebene zu beeinflussen. Hier steht jedoch in der Kritik, dass an wichtigen Entscheidungspunkten allzu offensichtlich wurde, dass dem Spieler nur vorgegaukelt wurde auch hier die Geschichte selbst beeinflussen zu k�nnen.\\
Im Gegensatz zu den eben genannten Arbeiten und Spielen soll in der aktuellen Arbeit au�erdem eine Analyse des Spielers erfolgen. Hier sollen bestimmte Eigenschaften seiner Spielweise anhand seines Verhaltens ermittelt werden. Die Ergebnisse werden anschlie�end genutzt, um die Manipulationen besser auf den Spieler abzustimmen und sie somit effektiver zu machen.
%, �hnlich der in \textit{Die Versuchung} stattfindenden, 
%w�hrend \textit{The Vanishing of Ethan Carter} viel mit visuellen Umgebungshinweisen arbeitete. Wobei jedoch vor allem bei Letzterem bem�ngelt wurde, dass am Ende des Spiels die geschaffene Illusion von Entscheidungsfreiheit aufgel�st wird, da alle nicht wahrgenommenen und auch unwichtigen R�tsel nachgeholt werden m�ssen. Dies soll im aktuellen Prototypen somit m�glichst vermieden werden.

%................................................................		% Verwandte Arbeiten einbinden
	%#####################################################################################################################
% Datei	: Konzept.tex
% Autor	: Dominique Kasper
%#####################################################################################################################
\chapter{Konzeptionierung}
Einleitung zu dem Konzept. 

%................................................................
\section{Manipulationen}
\label{sec:konzept manipulationen}

\section{Spielereigenschaften}
\label{sec:konzept spielereigenschaften}

\section{Level-Design}
\label{sec:level design}

\section{Zusammenfassung}
\label{sec:zusammenfassung konzeption}

%................................................................							% Konzeptionierung einbinden
	%#####################################################################################################################
% Datei	: Grundlagen.tex
% Autor	: Implementierung Worms
%#####################################################################################################################
\chapter{Prototypische Umsetzung}
Einleitung zu der Umsetzung. 

%................................................................
% Hier die Sektionen f�r die Umsetzung

%................................................................			% Implementierung einbinden
	%#####################################################################################################################
% Datei	: Evaluierung.tex
% Autor	: Byron Worms
%#####################################################################################################################
\chapter{Evaluierung}
\label{chap:evaluierung}

In diesem Kapitel soll zun�chst besprochen werden, wie gut die Auswertung der Eigenschaftenanalyse funktioniert hat. Die Probanden sollten gem�� den Eigenschaften Sammler, Geduld, Ehrgeiz, Herausforderung, Mitleid, Autonomie und Aufmerksamkeit beurteilt werden. Anschlie�end erfolgt die Auswertung des Interviews mit den Probanden, welche auch gezielte Fragen zu den einzelnen Manipulationen enthalten. Dabei werden positive Aspekte, Gr�nde und auch Verbesserungsm�glichkeiten vorgestellt. Abschlie�end wird auf die allgemeinen Fragen zum Spiel eingegangen und die Erfolgskriterien bewertet, sodass ein Fazit getroffen werden kann, wie erfolgreich insgesamt die verwendeten Manipulationen waren.

%................................................................
\section{Auswertung Eigenschaftenanalyse}
\label{sec:auswertung eigenschaftenanalyse}

\subsection{Fehlerquellen in der Umsetzung}
\label{subsec:fehlerquellen in der umsetzung}

Eine gro�e Fehlerquelle war die ung�nstige Platzierung der Trigger, die f�r die Analyse der Eigenschaften dienen sollten. Dies hatte zur Folge, dass manche Probanden falsch bewertet und die nachfolgenden Level nicht entsprechend korrekt aufgebaut wurden. Problematisch dabei war die Annahme, dass der Spieler immer linear weiter laufen und nicht vor- und zur�ckspringen bzw. beide Wege abdecken soll. So kann gleich bei dem ersten Diamanten im Spiel passieren, dass der Spieler zun�chst dar�ber springt und als Nicht-Sammler klassifiziert wird. Kehrt der Spieler dann aber zur�ck und sammelt den Diamanten ein, ist er offensichtlich doch ein Sammler. Bei der kleinen Kirsche, bei der der Spieler eine Aufgabe annehmen konnte, waren vier V�gel auf einem Plateau zu besiegen. Der Trigger wurde aber erst weiter hinten bei den Eltern der Kirsche gesetzt. Besiegt der Spieler die V�gel und geht nicht weiter zu den Eltern vor, l�st er den Trigger ggf. nicht aus. Eine weitere Problemstelle war der bewegliche Baumstamm unter den vielen kleinen Baumst�mmen. W�hlt der Spieler den oberen, schwierigeren Weg und f�llt dann runter auf dem Baumstamm, der ihn auf die andere Seite bringt, wird nicht das Nehmen des oberen, sondern das des unteren Weges gez�hlt. Die letzte Fehlplatzierung befand sich bei dem Pfeilschild, wo unter anderem die Geduld des Spielers getestet werden sollte, indem unter dem Baumstamm langsam V�gel hervorkamen. Der Trigger befand sich am Ende des unteren Baumstamms. Die meisten Spieler, die warteten, haben nur die V�gel besiegt und sich dann f�r den oberen Weg entschieden und haben so den vorgesehenen Trigger nicht erreicht. \\

Eine weitere Fehlerquelle ist die fehlende Sichtbarkeit einiger Punkte, die zur Verzerrung der Bewertung f�hrten. Dazu z�hlen zum Beispiel die Diamanten, die sich vor der gro�en Schlucht in einer kleinen Einbuchtung befinden. Auch die Kuhle mit den vielen V�geln wurde nicht als solches erkannt, sondern als eine Art Schlucht, �ber die man springen muss, um nicht zu sterben. Oftmals war es den Probanden nicht klar, dass der bewegliche Baumstamm auch zum Ziel f�hrt. Dieser war zun�chst mit dem Gedanken an 2D-Jump-and-Runs entworfen worden, wo der Querschnitt des Bodens ebenfalls sichtbar ist, sodass man den Baumstamm und das zu erreichende Ziel auch gleich erkennt. Am Ende des Levels ist die zu rettende Kirsche nicht mehr erreichbar, wenn der Spieler schon alle Eier zerst�rt hat, sodass auch bei Mitleid des Spielers diese Eigenschaft nicht mehr erfasst werden konnte. Auch die Hilfeschreie dieser Kirsche wurden von den Probanden eher als witzig und nicht als mitleiderregend aufgefasst. Daraus und dass der Schrei in manchen F�llen schlecht wahrnehmbar war, l�sst sich schlie�en, dass dies nicht der richtige Anreiz war, um Mitleid erzeugen zu k�nnen.

\subsection{Verbesserungsm�glichkeiten}
\label{subsec:verbesserungsm�glichkeiten}

Eine wichtige technische Verbesserung w�re, dass das Triggersystem adaptiv arbeiten muss. Es muss auch F�lle abdecken, in denen der Spieler r�ckw�rts l�uft oder beide Wege nimmt. Dies f�hrt allerdings auch zu mehr Komplexit�t.

\subsection{Folgen}
\label{subsec:folgen}

\subsubsection{Sammler}

Als Folge wurde das Sammeln meist falsch bewertet. Die meisten Probanden sammelten viel bis sehr viel, bekamen aber negative Bewertungen auf die Eigenschaft Sammeln. Positiv zu erw�hnen ist, dass die Spieler tats�chlich bem�ht waren, die Diamanten zu sammeln. Diese boten also einen entsprechenden Anreiz. Manche Probanden haben allerdings nicht mitbekommen, dass sie dadurch mehr Leben bekommen, haben aber trotzdem gesammelt. Der zu schnelle Text k�nnte daf�r ein Grund sein. Weitere Gr�nde f�r die falsche Auswertung der Sammel-Eigenschaft liegen auch an dem bereits schon angesprochenen ersten Diamanten, der nachtr�glich noch eingesammelt werden konnte, sowie an den Diamanten, die sich in der Schlucht befanden, aber von den Probanden nicht gesehen wurden. Diese Stellen zu verbessern, w�rde eine korrektere Auswertung nach sich ziehen. Noch besser w�re aber die Analyse mittels eines Schwellenwertes, bei dem die Anzahl der gesammelten Diamanten mit der Anzahl der im Spiel befindlichen Diamanten verglichen wird. So ist die Sammeleigenschaft nicht nur an den Triggern festgemacht.

\subsubsection{Geduld}

Die gr��te Abweichung in der Analyse betraf die Eigenschaft Geduld. Bei einigen Probanden stimmte die Analyse mit dem Fragebogen �berein. Bei anderen ergab sie Ungeduld, obwohl sich die Probanden selbst als geduldig eingestuft hatten. Hierbei ist zu erw�hnen, dass die Einsch�tzung der Geduld im Fragebogen sehr subjektiv erfolgte und es nicht sicher ist, ob sich der Spieler richtig einsch�tzen konnte. Als weiterer Grund fehlte ein zus�tzlicher Anreiz an entsprechenden Stellen (z.B. durch Diamanten), um auch wirklich geduldig sein zu m�ssen. Der von uns gew�hlte Ansatz, dass der geduldigere Weg angenehmer ist, da der andere Weg schwieriger oder durch V�gel blockiert ist, funktioniert nur, wenn der Spieler Feinde und schwere Wege scheut. Auch der bereits oben erw�hnte falsch gesetzte Trigger bei dem Warten auf die V�gel bzw. das falsche Deuten des sich bewegenden Baumstammes sind m�gliche Gr�nde der Fehlanalyse. Verbesserungen k�nnten einerseits durch besseres Setzen der Trigger und das Geben von mehr Anreizen f�r die Wahl des geduldigen Weges erzielt werden. 

\subsubsection{Ehrgeiz}

Eher geringe bis gar keine Abweichungen gab es bei der Eigenschaft, ob der Spieler den schweren Weg eher nimmt als den leichten. Dies f�hrt zu der Schlussfolgerung, dass der schwere Weg am Anfang auch tats�chlich als solcher wahrgenommen wurde. Kaum ein Proband hat ihn gew�hlt. Wenn doch, dann wegen der sich dort befindlichen Diamanten oder der Proband sah diese Strecke als Herausforderung an und w�hlte sie deshalb, was auch so vorgesehen war. F�r die geringen Abweichungen sei wieder auf das bereits erw�hnte Herunterfallen vom schweren Weg auf den sich darunter bewegenden Baumstammes verwiesen. Auch ist die Wahrnehmung, ob ein Weg schwer oder leicht ist, abh�ngig vom Spieler und k�nnte so die Abweichungen im Fragebogen erkl�ren.

\subsubsection{Herausforderung}  

Auch bei der Eigenschaft der Herausforderung gab es meist nur geringe Abweichungen bei der Analyse. Die Analyse hat also diesbez�glich allgemein gut geklappt. Es gab nur einen Ausrei�er, der bei der Anzahl an Probanden vernachl�ssigbar ist. Positiv war hierbei, dass die Stellen bei der kleinen Kirsche und �ber dem matschigen Weg gut umgesetzt wurden. Der Spieler hatte bei der Aufgabe der kleinen Kirsche jederzeit die Wahl, aus Angst vor der Herausforderung auch abbrechen zu k�nnen. Der �berwiegende Teil der Spieler hat sich als mehr Herausforderung suchend empfunden (im Fragebogen) als dies von der Analyse bewertet wurde. Ein Grund ist einmal die Kuhle mit vielen Feinden drin, die als solche nicht erkannt wurde. Ein weiterer Grund ist das lange Warten auf die V�gel und die darauf folgende Wahl, trotzdem oben lang zu gehen, sodass der richtige Trigger nicht ausgel�st wurde. Verbesserungsm�glichkeiten sind somit das Beheben des Triggerproblems, sowie die Verbesserung der Sichtbarkeit der Kuhle.  

\subsubsection{Mitleid}

Bei der Mitleids-Eigenschaft gab es einige Abweichungen bei der Analyse, sowohl zu mehr als auch zu weniger Mitleid. Positiv hervorzuheben ist, dass erreicht werden konnte, dass ein Proband tats�chlich Mitleid mit der kleinen Kirsche hatte. Einige Gr�nde f�r die Abweichungen wurden bereits angesprochen. Dazu z�hlen der Weg zur Kirsche, der durch zerst�rte Eier unm�glich wurde, der nicht als Hilferuf wahrgenommene Schrei und der sich erst bei den Eltern befindliche Trigger. Alle drei Situationen sind Gr�nde daf�r, dass jemand als weniger mitleidig eingestuft wurde, als er eigentlich ist. Wie mitleidig sich der jeweilige Spieler selbst einstuft ist zudem wieder sehr subjektiv. M�glicherweise wollten die m�nnlichen Probanden ihr Mitleid nicht im Fragebogen zugeben. Auch ist oft die eigene Erfahrung ausschlaggebend, ob sich eine Person in die jeweilig Situation hineinversetzen kann (z.B. wie die Kirsche selbst Eltern verloren, etc.). Dies k�nnen sowohl Gr�nde f�r mehr als auch f�r weniger analysiertes Mitleid sein. Generell ist es schwierig, sich mit derart prototypischen und abstrakten Charakteren zu identifizieren. Es gibt einige Verbesserungsm�glichkeiten, die in diesem Fall getroffen werden k�nnen. Ein realistischerer Sound f�r den Schrei der Kirsche k�nnte beim Probanden mehr Mitleid erzeugen. Dies war f�r den Prototypen schwierig, da kostenlose Sounds nur schwer zu bekommen sind. Und auch das Einsprechen eines solchen ber�hrenden Schreis erfordert ebenfalls Geschick, welches nicht jeder besitzt. Als weiterer Punkt kann die Kombination mit den Eiern verbessert bzw. vermieden werden. Da der Spieler vorab lernt, die Eier zu zerst�ren, ist es nicht zielf�hrend, diese f�r den Weg zur Kirsche zu verwenden. Wichtiger w�re gewesen, beide Eigenschaften einzeln zu testen (einmal Mitleid f�r die Kirsche und einmal Aufmerksamkeit durch die gl�nzenden Eier). Allerdings bleibt zu sagen, dass es eher schwer ist, subjektive und individuelle Abweichungen bei der Bewertung und Analyse des Mitleids zu umgehen. Es kann maximal versucht werden, die Charaktere realistischer zu gestalten und �ber eine l�ngere Zeit eine Beziehung zu ihnen aufzubauen, damit der Spieler eine parasoziale Beziehung eingeht. Dies ist jedoch im Rahmen eines kurzen Prototypen nicht m�glich. Auch m�sste die Story bunter und vielf�ltiger ausgebaut werden, damit sich der Spieler besser hineinf�hlen kann, was wiederum erst in fertigen Spielen wirklich m�glich ist.

\subsubsection{Autonomie} 

Bei der Autonomie gab es teilweise gr��ere Abweichungen. Die Spieler haben sich gr��tenteils autonomer eingesch�tzt als sie eigentlich analysiert wurden. Positiv ist, dass keine Fehler durch die Trigger zur Autonomiemessung auftauchten. F�r die Abweichung mag der Hauptgrund sein, dass die Spieler es gewohnt sind, den Anweisungen in einem Tutorial zu folgen. Wenn ihnen gesagt wird, dass sie Dinge ausprobieren sollen (zum Beispiel Besiegen eines Gegners), machen sie es vor allem, um erste Erfahrungen mit der Steuerung, etc. zu bekommen. Somit sollte die Autonomie au�erhalb des Tutorials getestet werden, indem zum Beispiel Antonio irgendwann auf einen Weg hinweist und Billy auffordert, diesen zu gehen.

\subsubsection{Aufmerksamkeit} 

Die Eigenschaft Aufmerksamkeit hatte die geringste mittlere Abweichung. Teilweise wurde der Spieler als weniger aufmerksam analysiert, meistens aber als aufmerksamer, wenn es �berhaupt eine Abweichung gab. Die Situationen, in denen die Aufmerksamkeit gemessen wurden, waren kaum falsch zu verstehen und durch Fehler zu umgehen, sodass dies als positiv zu werten ist. Dennoch gab es auch hier Schwankungen, da auch Aufmerksamkeit eher eine subjektive Eigenschaft ist. Es ist auch denkbar, dass die Eier nicht aufgrund des Leuchtens, sondern zwecks Belohnung zerst�rt wurden. Durch zwei Verbesserungsm�glichkeiten k�nnte die Schwankung verringert werden. Das Pfeilschild war eher auff�llig, auch wenn dies nicht unbedingt nach sich zog, dass der unsichtbare Baumstamm entdeckt wurde. Eine weniger auff�llige M�glichkeit h�tte zum Beispiel darin bestanden, eine kleine Pyramide dorthin zu stellen. Des weiteren wurde die Aufmerksamkeit in der Aush�hlung vor der Schlucht getestet. Da diese relativ schwer zu sehen und nur durch genaues Erkunden des Levels zu entdecken war, h�tte das eine h�here Anforderung an die Aufmerksamkeit des Spielers bedeutet.



\section{Auswertung Interview}
\label{sec:auswertung interview}

	
\subsubsection{1. Inwiefern hast du dich durch die Gespr�che mit den Charakteren beeinflusst gef�hlt? Gab es Unterschiede zwischen dem Level 2.1 und den anderen Leveln?}

Die Frage nach der Beeinflussung wurde sehr ausgewogen beantwortet. Ein Teil der Probanden hat sich beeinflusst gef�hlt, der andere nicht. Ein Grund daf�r ist einerseits die fehlende Beziehung zu den Charakteren. Dadurch fiel die Beeinflussung nicht sehr stark aus. Allerdings ist das auch wiederum eine Einstellungssache, da sich einige Spieler durchaus beeinflusst gef�hlt haben. Ein paar wenige hatten sogar Mitleid mit den Figuren. Ein weiterer Grund f�r die geringe Beeinflussung war der zu schnelle Text, aufgrund dessen die Spieler der Geschichte nicht ganz folgen konnten. Verbesserungen k�nnen durch den Ausbau des Prototypen bez�glich der Story und des Beziehungsaufbaus zu den einzelnen Figuren erreicht werden. Zwischen dem langen, auf die Eigenschaften ausgerichteten Gespr�ch und den anderen Gespr�chen, in denen nicht mehr zwischen den analysierten Eigenschaften differenziert wurde, wurden von den Probanden keine Unterschiede festgestellt. Nur ein Proband bemerkte, dass die Gespr�che l�nger bzw. detaillierter waren. Im Allgemeinen scheint die Differenzierung nach Merkmalen in den Gespr�chen bei den Spielern nicht bewusst aufgefallen zu sein. Dies kann als positiv gewertet werden, da so die versuchte Manipulation nicht bemerkt wurde. Auf der anderen Seite schien der Einfluss laut dem Interview nicht allzu gro� zu sein. Hierbei ist besonders die noch folgende Befragung der Probanden zu den einzelnen Quests zu beachten, da es schwer einzusch�tzen ist, was der wahre Grund f�r die Annahme der Quest war. Letztendlich haben die Spieler die Quests nicht abgelehnt, wodurch das angestrebte Ziel erreicht wurde und somit als Teilerfolg zu werten ist. \\

\textbf{Schlussfolgerungen f�r die Wirksamkeit der angepassten Gespr�che} \\

Nur zwei Spieler wurden als Herausforderung suchend eingestuft und haben beide die Quests in Level 2.1 und 3.1 angenommen. Das k�nnte bedeuten, dass die Anpassung des Gespr�chs keinen Einfluss hatte. Alle anderen Spieler bekamen den Text f�r Mitleid bzw. alle restlichen Eigenschaften zu sehen. Bis auf zwei davon haben alle Spieler sowohl in 2.1 als auch in 3.1 die Quests angenommen. In 2.1 lag die Begr�ndung daf�r bei der angepriesenen Belohnung. Diese war weitaus besser als die Belohnung in 3.1. Die Gespr�chsanpassung scheint ebenfalls unwirksam, da Mitleid nicht der Grund des Annehmens war, sondern die bessere Belohnung. Im Allgemeinen l�sst sich sagen, dass die Anpassung der Gespr�che weniger sinnvoll war, denn die Wirkung war nicht dem Aufwand entsprechend, sodass ein weiteres Mal darauf verzichtet werden kann. Es gelten hier dieselben Gr�nde wie bereits bei der Beeinflussung genannt. Eine Verbesserung kann man h�chstens durch den Ausbau des Prototypen erzielen, oder man schafft aufgrund des Aufwandes die Gespr�chsdifferenzierung generell ab. \\

\textbf{Fazit:} 
Die Anpassung der Gespr�che an die Eigenschaften der Spieler war nicht erfolgreich. 


\subsubsection{2. Findest du, dass du dich an einem Wegpunkt falsch entschieden hast?}

Diese Frage sollte dazu dienen, herauszufinden, ob die Manipulationen unauff�llig genug waren. Die meisten Spieler haben sich insofern ge�u�ert, dass sie einen alternativen Weg noch f�r m�glich gehalten h�tten bzw. sagten nicht explizit, dass sie ihn f�r unm�glich gehalten haben. Nur zwei Spieler haben entweder nicht gemerkt, dass es Alternativen gab oder gemerkt, dass der Entwickler nicht wollte, dass ein spezieller Weg genommen wird. Auf Verbesserungsm�glichkeiten bez�glich einzelner Manipulationen wird nachfolgend noch genauer eingegangen. \\

\textbf{Fazit:} 
Die verwendeten Manipulationen erf�llen das Kriterium der Unauff�lligkeit und der Wirksamkeit bei einer m�glichst gro�en Personenanzahl.


\subsubsection{3. Glaubst du, du h�ttest ein alternatives Ende erreichen k�nnen? Warum?}

Die meisten Spieler hielten kein alternatives Ende f�r m�glich. Ein Spieler konnte nicht gewertet werden, da es einen technischen Fehler ab dem Kampf mit Black Sparrow gab, weshalb das Spiel nicht zu Ende gespielt werden konnte. Mithilfe von komplexeren Leveln k�nnten noch mehr unterschiedliche Wege eingebaut werden, sodass es nicht einfach �berpr�fbar ist, ob es andere m�gliche Wege gegeben h�tte. Auch eine komplexere Story und Interaktionsm�glichkeiten (z.B. umfangreichere Gespr�che mit den Charakteren) kann zu einer Verbesserung f�hren. \\

\textbf{Fazit:} 
Es besteht kein gro�er Unterschied zwischen der Zahl von Spielern, die ein alternatives Ende f�r m�glich hielten und denen, die dies nicht taten. Somit ist davon auszugehen, dass vorhandene L�sungen durchaus Potenzial haben, die durch Umsetzung der Verbesserungsvorschl�ge sogar zu guten Ergebnissen f�hren k�nnten.


\subsubsection{4. Warum hast du Bob nicht geholfen? Wenn doch, warum?}

Viele Spieler gaben an, dass sie nicht genau wussten, was im zweiten Level zu tun ist. Dies kann zum Beispiel an dem zu schnellen Text liegen. Einige Probanden sagten, dass sie dadurch die Handlung nicht mitbekommen haben. Auch der Wegweiser und seine Funktion war den Spielern nicht bekannt. Au�erdem wurde von einigen Spielern Bob und die Interaktionsm�glichkeit mit ihm �bersehen. Verbesserungsm�glichkeiten liegen bei der Einstellung eines langsameren und deutlicherer erkennbaren Text. Auch k�nnte Bob den Spieler beim Vorbeigehen automatisch ansprechen, sodass der Interaktionshinweis wegfallen kann. Dies k�nnte allerdings wieder die Selbstst�ndigkeit des Spielers beeinflussen. Daher w�re eine andere M�glichkeit, den Spieler durch Rufen oder �hnliches auf Bob aufmerksam zu machen. Allgemein erscheint es besser, wenn die Figuren nicht durch Text, sondern mittels einer Stimme sprechen w�rden. Die Funktion des Wegweiser sollte schon am Anfang eingef�hrt werden, damit der Spieler wei�, wozu dieser gut ist. Noch besser w�re es allerdings, wenn der Wegweiser durch eine kleine Karte ersetzt werden w�rde, auf der sich der Spieler bewegen und dann das entsprechende Level betreten kann. Diese sollte au�erdem schon ab Billys Haus zug�nglich sein, um Verwirrung durch das automatische Weiterleiten zu 2.1 zu vermeiden. \\


Die urspr�ngliche Manipulation basierte auf \grqq Armes-Schwein-Spiel\grqq. Der Spieler sollte Mitleid mit Bob haben, da sein Dorf solche Schwierigkeiten hatte. Deshalb bekam auch jeder Spieler, es sei denn er war Herausforderung suchend, den Mitleidstext angezeigt. Als Ausweg war gedacht, dass der Spieler ohne das bei Bob erh�ltliche Schild nicht weiterkommt und so im Zweifelsfall gezwungen wird, dieses zu holen. Vier Spieler gaben letztendlich an, dass sie aus Hilfsbereitschaft/Verantwortungsgef�hl/ Mitleid geholfen haben. Aber der am h�ufigsten genannte Grund (der auch zweimal in Kombination mit Mitleid/Verpflichtung auftrat) war die Belohnung, die der Spieler bei Hilfe erhalten w�rde. Meist hat also die sekund�re Manipulation angeschlagen und war somit wirksamer als die prim�re, auch wenn einige auf diese eingingen. 50\% der als mitleidig eingestuften Spieler haben tats�chlich aus Mitleid gehandelt, die restlichen 50\% wegen der Belohnung. Einer der Nicht-Mitleidigen hat mitleidig gehandelt. Insgesamt scheint die Mitleidseigenschaft kaum Einfluss genommen zu haben, was wiederum unter anderem daran liegen kann, dass Bob als nicht sehr mitleiderregend wahrgenommen wurde. Die bereits vorgestellten Verbesserungsm�glichkeiten bzgl. der Gespr�che und dem Realismus des Prototypen w�ren hier wieder hilfreich.


\subsubsection{5. Denkst du, dass du Bob zu mehr Unterst�tzung h�ttest �berreden k�nnen, wenn du das Gespr�ch anders gef�hrt h�ttest?}

Ungef�hr die H�lfte der Probanden meinte, dass sie Bob h�tten �berreden k�nnen. Die andere H�lfte hielt dies f�r nicht m�glich. Und ein Proband gab eine Antwort, die nicht interpretierbar ist. Auch hier k�nnten zum Beispiel komplexere Gespr�che und Figuren zu einem besseren Ergebnis hinsichtlich des Manipulationsversuches f�hren. \\

\textbf{Fazit:}
Es kann keine eindeutige Aussage getroffen werden, da es weder in die eine noch die andere Richtung eine Tendenz gibt.


\subsubsection{6. (Spieler hat den Untergrund ignoriert) Hast du den zerst�rbaren Block gesehen? Wenn ja, wieso hast du dich dagegen entschieden?}

Fast alle Spieler haben den versteckten Eingang nicht bemerkt. Ein Proband hat ihn zu sp�t gesehen, sodass dieser bereits eingest�rzt war. Daraus folgt, dass die Umsetzung zu unauff�llig war. Die zerst�rbaren Bl�cke m�ssten zu einem fr�heren Zeitpunkt des Levels eingef�hrt werden, damit der Spieler mit dem Prinzip bereits vertraut ist. Dar�ber hinaus m�sste der Block besser erkenntlich gemacht werden, damit er nicht so leicht zu �bersehen ist, zum Beispiel durch eine andere Farbe, Struktur oder �hnliches. \\

\textbf{Fazit:}
Aufgrund der Tatsache, dass der gew�nschte Bereich nicht wahrgenommen wurde, l�sst sich auch hier keine Einsch�tzung zu der Manipulation treffen.


\subsubsection{7. (Spieler hat den Untergrund nicht ignoriert) Warum hast du dich nicht f�r den oberen Weg entschieden?}

Es k�nnen keine Aussagen gemacht werden, da kein Spieler diesen Weg erreicht hat.


\subsubsection{8. Warum hast du dich f�r diesen (Br�cken-)Weg entschieden? H�ttest du lieber den anderen Weg genommen?}

F�nf Probanden haben nicht mitbekommen, dass der rechte Weg wieder auf den Hauptweg zur�ckf�hrt. Sechs Probanden haben den rechten Weg nicht gesehen. Davon wurden alle durch den Vogel unter der Br�cke gezwungen, den ersten Weg zu nehmen. Ein zus�tzlicher Spieler hat zwar den rechten Weg gesehen, wurde aber ebenfall durch den Vogel nach oben gezwungen. Zwei Probanden haben gleich den rechten Weg genommen. Einer davon gleich beim ersten Mal, der andere erst im Nachhinein. Letzterer hat au�erdem mitbekommen, dass beide Wege zum gleichen Ziel f�hren, w�hrend ersterer dies nicht tat. Somit sollte als Verbesserung erst einmal der Vogel nicht unter die Br�cke gesetzte werden. Die Spieler waren so nicht in der Lage die zweite Wegm�glichkeit zu sehen, wodurch die Ablenkung von den vielen Manipulationen nicht so gut wie geplant erfolgen konnte. Der Spieler f�hlte sich au�erdem durch den Vogel gen�tigt, die linke Br�cke zu nehmen. Zudem h�tte das Einf�hren des Spielelements Br�cke schon fr�her erfolgen m�ssen, damit kein Spieler \grqq aus versehen\grqq \ die Br�cke hochspringt, ohne dass er diesen Weg nehmen wollte. Besonders die Kombination mit dem Vogel war daher sehr ung�nstig. \\

\textbf{Fazit:}
Die Ablenkung von den Manipulationen konnte nicht ihre volle Wirkung entfalten, da die meisten Spieler den alternativen Weg nicht sahen und/oder vom Vogel auf den linken gezwungen wurden.


\subsubsection{9. (Spieler ist gleich rechts lang) Warum hast du den linken Weg unbeachtet gelassen? }

Fast alle Spieler wurden von dem schwierigen Weg abgeschreckt. Einer von ihnen hat mitbekommen, dass dies so war, weil der Entwickler nicht wollte, dass der Spieler den Weg nimmt. Nur ein Proband hat den schweren Weg genommen, weil er etwas gro�es dahinter vermutete. Vier Spieler sind zur sekund�ren Manipulation gelangt. Nur einer davon hat gemerkt, dass diese nicht zu �berbr�cken war. Die anderen dachten, dass es irgendwie m�glich w�re, aber sie noch keinen geeigneten Weg zur �berbr�ckung gefunden haben. Nur ein Spieler hat den Schrei bemerkt und ihn mit in seine Entscheidung einbezogen, sodass sich sagen l�sst, dass diese zus�tzliche Manipulation nicht wirksam war. Der Schrei war m�glicherweise zu unauff�llig (zu leise, wiederum keine Sprache).  \\

\textbf{Fazit:}
Insgesamt wurden die Manipulationen kaum durchschaut und sind dadurch erfolgreich zum Abschrecken vieler Personen, w�hrend dieses nicht auf die Manipulation durch den Schrei zutrifft.


\subsubsection{10. (Spieler ist bei Vogelschwarm nicht aus dem Nest raus) Warum hast du das Nest nicht durch die T�r verlassen?}

So gut wie alle Spieler sind aus der T�r raus als der Vogelschwarm kam. Nur ein Spieler ist nach links gerannt, weil er die T�r nicht als solche erkannt hatte. An dieser Stelle sollte besser kenntlich gemacht werden, dass es sich um einen gef�hrlichen Vogelschwarm handelt, der nicht besiegt werden kann, um den Spieler nicht zu verwirren. Zus�tzlich w�rde ein Feedback beim Tod durch den Vogelschwarm helfen. \\

\textbf{Fazit:}
Trotz der technischen/grafischen M�ngel war die Manipulation f�r die Zwecke des Prototypen ausreichend und hat die meisten Spieler dazu gebracht, das Nest zu verlassen.


\subsubsection{11. (Spieler hat die Kiwano ignoriert) Warum hast du die Kiwano nicht gerettet? Und wenn doch, warum?}

Ungef�hr die H�lfte der Spieler wurde durch den Schrei veranlasst, der Kiwano zu helfen. Ein Proband war allerdings vom Schrei genervt. Der Schrei wurde im Zusammenhang mit dem Charakter als Zeichen daf�r gesehen, dass geholfen werden muss. Der Schrei alleine scheint nicht auszureichen.
Fast alle Probanden haben der Kiwano geholfen bzw. wollten ihr helfen.  Drei Probanden gaben an, geholfen zu haben, weil sie sowieso schon da waren. \\

\textbf{Fazit:}
Im Vergleich zu der vorherigen Situationen, wo nur der Schrei zu h�ren war, ist davon auszugehen, dass die Unauff�lligkeit des Schreis ein entscheidender Faktor ist, da haupts�chlich im Zusammenhang mit Sicht auf die schreiende Figur geholfen wurde. Dies wird dadurch best�tigt, dass ein paar wenige Probanden den Schrei nicht zuordnen konnten (sowohl in 1.1 als auch in 2.1). Ansonsten scheint diese Kombination aus Sicht und Schrei den meisten Spielern zu suggerieren, dass sie helfen m�ssen/sollen.


\subsubsection{12. Warum bist du in die H�hle gegangen? Wenn nicht, warum nicht?}

Mehr als die H�lfte der Spieler sind gleich in die H�hle gegangen, der Rest w�hlte den Weg nach oben, wovon zwei Spieler sp�ter noch die H�hle erkundet haben. Dadurch bekamen diese beiden Probanden mit, dass beide Wege zum selben Ziel f�hrten. Keiner der Spieler gab an, wegen des Scheins aus der H�hle diesen Weg gew�hlt zu haben. Vier Probanden empfanden den Weg nach oben schwerer, einer davon hatte ihn genau deswegen genommen, die anderen drei vermieden. Zwei Probanden haben die Eier zerst�rt, sodass sie den Weg nach oben nicht mehr nehmen konnten. Einer davon wurde als unaufmerksam gewertet, der andere empfand den Weg nach oben als zu schwer. Die H�hle wurden von zwei Spielern sogar als gef�hrlich eingestuft. Und wiederum zwei Spieler haben erst gar nicht ihre Aufmerksamkeit nach oben gerichtet. Der Schein der H�hle sollte die Spieler anlocken. Anscheinend wurde dieser aber zu unauff�llig dargestellt, sodass der Unterschied zwischen unterem und oberen Weg nicht gro� genug war. Deshalb sollte f�r eine Verbesserung ein dominanterer visueller Umgebungshinweis als der Schein verwendet werden bzw. der Schein sich z.B. durch eine andere Farbe auff�lliger von der Umgebung abheben. Auch der Weg nach oben sollte schwieriger zu entdecken sein. Die Stufen k�nnten dem Spieler assoziiert haben, dass er sich nach oben begeben soll, zumal dieser ein ganzes St�ck vor der H�hle angefangen hat.  \\

\textbf{Fazit:}
Das Konzept des Zusammenspiels der Manipulationen funktionierte potenziell gut. Der Weg nach oben wurde zum Beispiel durch die Eier erschwert. Diese konnten durch Unaufmerksamkeit zerst�rt werden und die H�hle wurde durch Diamanten interessanter.
Die Manipulation durch den Schein verlief allerdings besonders schlecht.


\subsubsection{13. Warum hast du den Hut am Anfang nicht aufgenommen? Und wenn doch, warum?}

Nur vier Probanden entschieden sich daf�r, den Hut nicht aufzunehmen. Davon haben zwei Probanden ihn nicht erkannt. Ein weiterer Proband wollte sich von diesem nicht ablenken lassen und der letzte fand ihn generell einfach uninteressant. Die restlichen Spieler haben den Hut aufgrund von Neugier, aufgrund des Questgedankens oder einfach weil das Interface es so suggeriert hat, aufgenommen. Um noch mehr Probanden zur Aufnahme des Hutes zu motivieren, k�nnte dieser auff�lliger gestaltet werden, zum Beispiel durch Leuchten oder anderweitigem Abgrenzen von der Umgebung. \\

\textbf{Fazit:}
Ein Fazit kann erst gezogen werden, wenn die dazugeh�rige Quest behandelt wird. Nachdenkbar w�re hier allerdings, die Neugier der Leute bei solchen Sachverhalten zum Zwecke weiterf�hrender Manipulationen auszunutzen (zum Beispiel zum Leiten, �hnlich wie bei den Diamanten).


\subsubsection{14. Warum hast du die Quest ignoriert? Wenn nicht, warum nicht?}

Nur zwei Probanden haben die Quest ignoriert, weil sie kein Mitleid mit Manfred hatten. Die restlichen nahmen die Quest wegen der Belohnung, aus Mitleid oder auch aus Neugier an. F�nf Probanden haben allerdings die Quest nicht beendet, einerseits weil sie vorher in den Zustand der Vogelverfolgung gerieten und andererseits weil es vergessen wurde. Positiv hier zu erw�hnen ist, dass bis auf zwei Spieler alle die Quest annahmen, davon sagte auch nur einer, dass er es tat, weil er eh schon den Hut hatte. Bei einer �berarbeitung k�nnten die Objekte auff�lliger gestaltet werden, um diese besser finden zu k�nnen, da es ein paar Aussagen diesbez�glich gab. Au�erdem sollte verhindert werden, dass der Spieler unbewusst vor Beendigung der Quest weiterkommt. \\

\textbf{Fazit:}
Die Manipulation hat gut funktioniert, wenn auch m�glicherweise aus den falschen Gr�nden (Belohnung und Mitleid gr��tenteils). Aber es ist nicht eindeutig auszuschlie�en, dass sich nicht schon ein in Besitz befindlicher Hut auch ausgewirkt hat.


\subsubsection{15. (Spieler vom Weg abgewichen) Warum bist du den V�geln nicht gefolgt?}

Viele Spieler haben nicht den Sinn der Vogelszene verstanden, denn sie wussten nicht, dass sie ihnen h�tten folgen sollen. Hierbei ist jedoch zu beachten, dass bei sechs Spielern Bugs auftraten, durch die f�nf von ihnen nicht wussten, was zu tun ist. Auff�llig dabei ist, dass bei allen, die verstanden haben, was zu tun ist, auch keine Bugs mehr auftraten. Nach �berarbeitung waren die V�gel n�mlich sichtbar und Antonio hat darauf hingewiesen, ihnen zu folgen. Von den Probanden sind zehn vom Weg abgewichen. Sie gaben an, die Umgebung des Weges zu interessant zu finden und versuchten deshalb z.B. noch schnell alle Diamanten einzusammeln. Au�erdem sind einige den Weg nach oben gegangen, weil sie entweder den Weg nach unten als t�dliche Schlucht ansahen oder weil sie dachten, dass die V�gel nach oben fliegen. Nur drei Probanden sind dem Pfad von Anfang an gefolgt, weil sie sich unter anderem von den V�geln gezwungen f�hlten, nicht abzuweichen. Insgesamt l�sst sich feststellen, dass die Vielzahl von Bugs die Wirkung sehr verf�lscht hat. Die Verfolgungssituation war nicht offensichtlich genug, die V�gel zu schlecht sichtbar und der unerw�nschte Weg war deutlich attraktiver als der gew�nschte. Diese Gr�nde sollten deshalb behoben werden, indem die Verfolgung besser eingeleitet und offensichtlicher gemacht wird, der Weg weniger attraktiv gestaltet wird und auch die V�gel von Anfang an sichtbar sind. \\

\textbf{Fazit:}
Es ist kaum eine Aussage �ber die Qualit�t der eigentlichen Manipulation m�glich, da der Sachverhalt zu sehr durch Bugs verzerrt wurde, aber vor allem der unerw�nschte Weg zu attraktiv im Gegensatz zum erw�nschten ausgearbeitet wurde. Es ist aber abzusehen, dass weitere Tests mit dem Konzept sinnvoll sind, da Spieler, die keine Bugs hatten, die Situation besser bew�ltigen konnten.


\subsubsection{16. Warum hast du den Bruder gesucht? Wenn nicht, warum?}

Weniger als die H�lfte der Spieler hat die Quest nicht angenommen, weil sie entweder zu sehr auf Cherry fixiert waren und nicht abweichen wollten oder einfach neugierig waren, was denn passieren w�rden, wenn abgelehnt wird. Die restlichen Probanden nahmen die Quest an, weil sie Antonio als einen Freund oder Begleiter empfanden und ihm deshalb helfen wollten oder weil sie sich dadurch weitere Unterst�tzung erhofft hatten. Als Verbesserung k�nnte generell besser erkenntlich gemacht werden, dass Antonio dem Spieler mit Herzen und Schilden hilft. Antonio muss ein eigenst�ndigerer Charakter werden und mehr interagieren bzw. mitk�mpfen und nicht nur an Billy h�ngen, damit die Spieler besser bemerken, dass er ihnen hilft.. F�r einen Prototypen w�re diese Umsetzung allerdings zu komplex. \\

\textbf{Fazit:}
Das Konzept der Manipulation scheint erfolgsversprechend bei besserer Umsetzung des Charakters von Antonio zu sein. Viele Spieler haben ihn schon allein wegen der Begleitung und den bisher vorhandenen Interaktionen als Begleiter angesehen, was ihn ihnen n�her brachte.


\subsubsection{17. Warum wolltest du Antonio trotz des Verrats nicht angreifen? Und wenn doch, warum?}

F�r den Angriff auf Antonio haben sich mehr als die H�lfte der Probanden entschieden. Davon wollten f�nf ihn f�r seinen Verrat bestrafen. Die restlichen gaben an, sowieso keine Beziehung zu ihm aufgebaut zu haben. F�nf Spieler wollten Antonio aus folgenden Gr�nden nicht angreifen. Entweder wollten sie die Entscheidung Cherry �berlassen, weil sie hofften, dass dadurch der Streit anders geschlichtet werden kann oder sie hofften generell, ein alternatives Ende erreichen zu k�nnen. Ein Proband gab an, eine zuf�llige Entscheidung getroffen zu haben. Bei dieser Manipulation k�nnte der Verrat noch schwerwiegender gestaltet werden, um noch mehr Spieler zum Angriff zu bewegen. Allerdings liegt die Schwere des Verrats wiederum beim Betrachter, weshalb manche reagierten und manche nicht. \\

\textbf{Fazit:}
Dieses Konzept der Manipulation ebenfalls erfolgsversprechend, da die Mehrzahl der Spieler Antonio angegriffen hat. Au�erdem hielten die, die nicht angegriffen haben, gr��tenteils ein anderes Ende f�r m�glich. Auch bei den angreifenden Spielern war nicht auszuschlie�en, dass sie ein alternatives Ende f�r m�glich halten. Insgesamt waren die  Manipulationen nicht zu auff�llig.





\section{Auswertung der Erfolgskriterien und der allgemeinen Fragen}
\label{sec:auswertung der erfolgskriterien und der allgemeinen fragen}


\subsubsection{Erfolgskriterien}

Die vorab gesetzten Erfolgskriterien konnten mit dem Prototypen erf�llt werden. 
Zun�chst gaben 85\% der Spieler an, Spa� beim Spielen gehabt zu haben. Von den Probanden h�tten 77\% bei erneutem Durchlauf einen anderen Weg gew�hlt. Der Mittelwert lag dabei bei 4.08 (von 5), wodurch davon ausgegangen werden kann, dass die Manipulationen unauff�llig genug waren, um den Spielern nicht unangenehm aufzufallen. 61\% der Spieler hielten ein anderes Ende f�r m�glich. Hierbei lag der Mittelwert bei 3.69, was darauf hindeutet, dass die meisten Spieler gemerkt haben, dass kaum ein anderes Ende m�glich ist. Allerdings konnten sie es auch nicht ganz ausschlie�en. Ein Proband hatte die Fragestellung falsch verstanden. Er dachte, dass die zuk�nftige Implementierung eines anderen Endes m�glich ist. Z�hlt man diesen Probanden mit, kommt es zu einer Verschlechterung (3.38), was allerdings das Ergebnis nicht wesentlich beeinflusst, da ein alternatives Ende insgesamt immer noch f�r m�glich gehalten wird. Insgesamt hielten damit 92\% der Probanden ein alternatives Ende oder einen alternativen Weg f�r m�glich, was auch dieses gesetzte Erfolgskriterium erf�llt. Bei dem Erfolgskriterium f�r das Nehmen der gew�nschten Wege l�sst sich prinzipiell sagen, dass alle Manipulationen von 3D auf 2,5D-Jump-and-Runs �bertragbar sind. Das Ergebnis l�sst sich folgenderweise einteilen:

\begin{itemize}
	\item Bereits gut funktionierte bzw. erfolgsversprechende Manipulationen:
	\begin{itemize}
		\item Wegentscheidung am Nesteingang des zweiten Levels und Schr�ge
		\item Vogelschwarm im Nest
		\item Hut-Quest
		\item Verlorener Bruder
		\item Verrat von Antonio
	\end{itemize}
	\item Schlecht funktionierte Manipulationen:
	\begin{itemize}
		\item An Spielereigenschaften angepasste Gespr�che
		\item \grqq Armes-Schwein-Spiel\grqq \ bei Bob
		\item Zerst�rbarer Block vor Untergrund
		\item Umgebungshinweise (Kiwano, H�hle)
		\item Verfolgung der V�gel
	\end{itemize}
	\item Nicht bewertbar:
	\begin{itemize}
		\item Manipulation im Untergrund, da niemand diesen erreicht hat
	\end{itemize}
\end{itemize}

Viele Manipulationen sind an die Beschr�nkungen (Audio, Charaktere, usw.) des Prototypen und Bugs (insbesondere das Folgen der V�gel) gesto�en. Wenn diese behoben werden,  ist jedoch zu erwarten, dass diese funktionieren. Die Analyse der Spielereigenschaften ist vom Ansatz her in Ordnung und auch sinnvoll, muss jedoch dringend erweitert werden. Da die Gespr�che zu wenig differenziert waren, hatten diese auch keinen sonderlichen Einfluss auf den weiteren Verlauf. Das Ergebnis f�r die Umgebungshinweise best�tigt das Ergebnis des Papers \grqq Player Manipulation\grqq, wobei der visuelle Umgebungshinweis m�glicherweise an der Umsetzung gescheitert ist. \\

Insgesamt kann der Prototyp als erfolgreich bezeichnet werden, da einige Manipulationen auch trotz der Beschr�nkungen gut funktioniert haben und auch die zuvor festgelegten Erfolgskriterien in Bezug auf Spa� und Unauff�lligkeit der Manipulationen erf�llt wurden. Das Erfolgskriterium f�r das Nehmen der gew�nschten Wege betrug allerdings nur 77\% aufgrund der Bugs beim Verfolgen der V�gel, w�rde aber ansonsten bei 81\% liegen und somit auch dieses Erfolgskriterium erf�llen. Die Ergebnisse haben auch gezeigt, dass das Verh�ltnis zwischen positiven und negativen Manipulationen, wie in der Konzeption gefordert, angemessen war, da es kaum zu Frustrationen seitens der Probanden kam.


\subsubsection{Allgemeine Fragen}

Der Mittelwert bei der Identifizierung mit den Charakteren liegt bei 2.85, was die bereits vorher gemachte Einsch�tzung dahingehend best�tigt, dass die Charaktere aufgrund der prototypischen Umsetzung zu abstrakt ausgefallen sind. Die Sympathie f�r Antonio erlangte den niedrigsten Mittelwert mit 2.46. Positiv war dabei der Ansatz zur Interaktion zwischen Antonio und Billy. Die geringe Sympathie basiert darauf, dass kaum ein Spieler mitbekam, dass Antonio Billy Hilfe in Form von Herzen und Schildern zukommen l�sst. Somit konnte diese Methode kaum ihre Wirkung entfalten. Auch hier ist ein Grund die sehr prototypische Umsetzung. Zudem war der gesprochene Text sehr schnell, sodass viele Probanden kaum mitkamen und so auch den Inhalt nicht voll erfassen konnten. Um bessere Werte zu erzielen, m�sste Antonio mehr mit Billy interagieren, sodass eine Beziehung aufgebaut werden kann. Der gesprochene Text muss langsamer und auch auff�lliger dargestellt werden. Au�erdem sollte die Unterst�tzung durch Antonio besser zu erkennen sein, indem der Spieler auf die Handlung an sich besser fokussiert wird oder bessere Grafik- und Soundeffekte gew�hlt werden. Der Mittelwert zur Frage, ob die Story des Spiels gefallen hat, lag bei 4.15. Dies zeigt, dass der Ansatz des Spielkonzepts Potenzial hat und dieser bei einem Ausbau �ber den Prototypen hinaus durchaus entfaltet werden kann. Eine zus�tzliche Best�tigung dazu liefert der Mittelwert von 4.31 bei der Frage nach dem Potenzial des Spiels bei einer Weiterentwicklung. Der Ansatz bei einem Ausbau erscheint f�r die Probanden sehr erfolgsversprechend.

%................................................................					% Evaluierung einbinden	
	%#####################################################################################################################
% Datei	: Zusammenfassung.tex
% Autor	: Byron Worms
%#####################################################################################################################
\chapter{Zusammenfassung und Ausblick}
Einleitung zu der Zusammenfassung. 

%................................................................
% Hier die Sektionen f�r die Zusammenfassung

%................................................................			% Zusammenfassung einbinden	
%................................................................
	\printbibliography[title={Literaturverzeichnis}]
	%#####################################################################################################################
% Datei	: 	Anhang.tex
% Autor	:	Byron Worms
%#####################################################################################################################
\begin{appendices}
\chapter{Abbildungen}
\label{chap:anhang_abbildungen}
\hideInContents
% Abbildung: Screenshots
\section{Screenshots}
%---------------------------------------------------------------------------------------------------------------------
\subsection{Hauptmen�}
\begin{figure}[H]
	\begin{center}
		\fbox{\includegraphics[page=1,width=0.98\linewidth]{Inhalt/Anhang/Screenshots/screenshot_0.png}}
	\end{center}	
	\caption[Hauptmen�]{Hauptmen� (Eigene Darstellung)}
	\label{fig:anhang_screenshots_menu}
\end{figure}
%---------------------------------------------------------------------------------------------------------------------
\subsection{Analyse--Level}
\begin{figure}[H]
	\begin{center}
		\fbox{\includegraphics[page=1,width=0.98\linewidth]{Inhalt/Anhang/Screenshots/screenshot_1.png}}
	\end{center}	
	\caption[Am Fu�e des Baums]{Am Fu�e des Baums (Eigene Darstellung)}
	\label{fig:anhang_screenshots_level1_baum}
\end{figure}
%
\begin{figure}[H]
	\begin{center}
		\fbox{\includegraphics[page=1,width=0.98\linewidth]{Inhalt/Anhang/Screenshots/screenshot_4.png}}
	\end{center}	
	\caption[Versteckter Weg]{Versteckter Weg (Eigene Darstellung)}
	\label{fig:anhang_screenshots_level1_weg}
\end{figure}
%---------------------------------------------------------------------------------------------------------------------
\subsection{Der kaputte Brunnen}
\begin{figure}[H]
	\begin{center}
		\fbox{\includegraphics[page=1,width=0.98\linewidth]{Inhalt/Anhang/Screenshots/screenshot_8.png}}
	\end{center}	
	\caption[Unterhaltung mit Bob]{Unterhaltung mit Bob (Eigene Darstellung)}
	\label{fig:anhang_screenshots_level2_bob}
\end{figure}
\begin{figure}[H]
	\begin{center}
		\fbox{\includegraphics[page=1,width=0.98\linewidth]{Inhalt/Anhang/Screenshots/screenshot_9.png}}
	\end{center}	
	\caption[Pfad mit Hindernissen]{Pfad mit Hindernissen (Eigene Darstellung)}
	\label{fig:anhang_screenshots_level2_hindernisse}
\end{figure}
\begin{figure}[H]
	\begin{center}
		\fbox{\includegraphics[page=1,width=0.98\linewidth]{Inhalt/Anhang/Screenshots/screenshot_12.png}}
	\end{center}	
	\caption[Eingest�rzte H�hle]{Eingest�rzte H�lle (Eigene Darstellung)}
	\label{fig:anhang_screenshots_level2_hoehle}
\end{figure}
%---------------------------------------------------------------------------------------------------------------------
\subsection{Der dunkle Wald} 
\begin{figure}[H]
	\begin{center}
		\fbox{\includegraphics[page=1,width=0.98\linewidth]{Inhalt/Anhang/Screenshots/screenshot_22.png}}
	\end{center}	
	\caption[Unterhaltung mit Manfred]{Unterhaltung mit Manfred (Eigene Darstellung)}
	\label{fig:anhang_screenshots_level3_manfred}
\end{figure}
\begin{figure}[H]
	\begin{center}
		\fbox{\includegraphics[page=1,width=0.98\linewidth]{Inhalt/Anhang/Screenshots/screenshot_24.png}}
	\end{center}	
	\caption[Verfolgung der Handlanger]{Verfolgung der Handlanger (Eigene Darstellung)}
	\label{fig:anhang_screenshots_level3_verfolgung}
\end{figure}
%---------------------------------------------------------------------------------------------------------------------
\subsection{Black Sparrow's Nest}
\begin{figure}[H]
	\begin{center}
		\fbox{\includegraphics[page=1,width=0.98\linewidth]{Inhalt/Anhang/Screenshots/screenshot_30.png}}
	\end{center}	
	\caption[Antonios Gest�ndnis]{Antonios Gest�ndnis (Eigene Darstellung)}
	\label{fig:anhang_screenshots_level32_antonio}
\end{figure}
\showInContents	
\end{appendices}								% Anhang einbinden	
\end{document}