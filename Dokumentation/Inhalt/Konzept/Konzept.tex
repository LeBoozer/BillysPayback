%#####################################################################################################################
% Datei	: Konzept.tex
% Autor	: Dominique Kasper
%#####################################################################################################################
\chapter{Konzeptionierung}
\label{chap:konzeptionierung}

Aufbauend auf den in Kapitel \ref{chap:grundlagen} und \ref{chap:verwandte arbeiten} erlangten Erkenntnissen lassen sich nun die für den Prototypen relevanten Manipulationen und Spielereigenschaften ableiten. Letztere dienen dabei zur Unterstützung der Wirksamkeit der angewandten Manipulationen.\\
Anschließend werden die Ergebnisse dieser Untersuchungen genutzt, um den entsprechenden Aufbau der Level auszuarbeiten. Diese sollen dazu dienen im Rahmen der abschließenden Evaluierung herauszufinden, inwiefern die verwendeten Manipulationen für den Einsatz in einem Jump and Run geeignet sind sowie 

%................................................................
\section{Manipulationen}
\label{sec:konzept manipulationen}

\section{Spielereigenschaften}
\label{sec:konzept spielereigenschaften}

\section{Level-Design}
\label{sec:level design}

\section{Zusammenfassung}
\label{sec:zusammenfassung konzeption}

%................................................................