%#####################################################################################################################
% Datei	: Konzept.tex
% Autor	: Dominique Kasper
%#####################################################################################################################
\chapter{Konzeptionierung}
\label{chap:konzeptionierung}

Aufbauend auf den in Kapitel \ref{chap:grundlagen} und \ref{chap:verwandte arbeiten} erlangten Erkenntnissen lassen sich nun die f�r den Prototypen relevanten Manipulationen und Spielereigenschaften ableiten, wobei letztere zur Unterst�tzung der Wirksamkeit der angewandten Manipulationen dienen.\\
Anschlie�end werden die Ergebnisse dieser Analyse genutzt, um den entsprechenden Aufbau der Level auszuarbeiten. So ist zun�chst ein Analyse-Level zu entwerfen, in welchem der Spieler auf die zuvor festgelegten Eigenschaften hin untersucht werden soll, indem sein Verhalten beobachtet und ausgewertet wird. Nachfolgend erfolgt der Entwurf der restlichen Level, in denen die theoretischen Konzepte einem praktischen Test unterzogen werden sollen.

%................................................................
\section{Manipulationen}
\label{sec:konzept manipulationen}

\subsection{Anforderungen}
\label{subsec:anforderungen manipulationen}

Bevor eine Auswahl an geeigneten Manipulationen getroffen werden kann, ist es zun�chst einmal notwendig herauszustellen, welche Anforderungen f�r eine solche erf�llt sein m�ssen. Diese lassen sich haupts�chlich aus den in Kapitel \ref{chap:grundlagen} beschriebenen Erkenntnissen �ber die Voraussetzungen f�r eine erfolgreiche Manipulation herleiten:

\begin{itemize}
	\item M�glichst Verwendung positiver Manipulationsstrategien
	\item Unauff�lligkeit
	\item Wirksamkeit bei m�glichst gro�er Menge von Pers�nlichkeiten
\end{itemize}

Das Kriterium der Unauff�lligkeit folgt hierbei implizit aus der Tatsache, dass ein wesentlicher Grundzug der Manipulation darin besteht die wahren Handlungsgr�nde des Manipulierenden zu verschleiern und somit den Interaktionspartner in seinen Entscheidungsm�glichkeiten einzuschr�nken. Aus diesem Grund ist anzunehmen, dass die Erfolgswahrscheinlichkeit einer Manipulation umso h�her ist desto schwerer es f�llt deren wahren Ziele zu erkennen.\\

\subsection{Einschr�nkungen}
\label{subsec:einschr�nkungen manipulationen}

Neben den Anforderungen sind auch einige Einschr�nkungen zu beachten. Diese sind sowohl durch das Spielgenre des Jump and Runs als auch durch  die begrenzten Ressourcen (z.B. Entwicklungszeit) f�r die Entwicklung des Prototypen bedingt. Auf diese Art und Weise k�nnen Manipulationen mit folgenden Eigenschaften nicht zum Einsatz kommen:

\begin{itemize}
	\item Aufbau des Manipulationseffekts �ber eine lange Zeitspanne
	\item Ansprache bestimmter Emotionen des Spielers (z.B. Mitleid) durch Mimik, Gestik oder Sprache der Charaktere
	\item Einsatz der Theorie des sozialen Vergleichs
	\item Verwendung von Questtageb�chern, um den Cliffhanger- bzw. Zeigarnik-Effekt zu erreichen
\end{itemize}

W�hrend die ersten drei Punkte eher durch fehlende Ressourcen bei der Prototypenentwicklung bedingt werden, ergibt sich die vierte Einschr�nkung aus der Definition des Jump and Run Genres, da in diesem traditionell keine Questtageb�cher vorgesehen und eher f�r andere Genres wie z.B. Rollenspiele typisch sind. Somit w�rde hier mit dem Einsatz eines solchen eine zu gro�e Abweichung von dem zu untersuchenden Spielgenre erfolgen.

\subsection{Ergebnisse}
\label{subsec:ergebnisse manipulationen}

Basierend auf den zuvor beschriebenen Anforderungen und Einschr�nkungen bez�glich geeigneter Manipulationen konnte schlie�lich eine Auswahl getroffen werden, die im Folgenden erl�utert werden soll.\\

\textbf{Armes-Schwein-Spiele}

%Auswahl noch nicht vollst�ndig, sondern noch in Arbeit

\section{Spielereigenschaften}
\label{sec:konzept spielereigenschaften}

\section{Level-Design}
\label{sec:level design}

\section{Zusammenfassung}
\label{sec:zusammenfassung konzeption}

%................................................................