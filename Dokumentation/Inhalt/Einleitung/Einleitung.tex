%#####################################################################################################################
% Datei	: Einleitung.tex
% Autor	: Ulrike Uderhardt
%#####################################################################################################################
\chapter{Einleitung}
\label{chap:einleitung}

Schon vor Jahrzehnten haben Computerspiele in die Kinder- und Wohnzimmer unserer Welt Einzug genommen und sind heutzutage kaum noch wegzudenken. Dabei dienen Computerspiele haupts�chlich der Unterhaltung der Spieler. Die Vielfalt an unterschiedlichen Spiele-Genre ist enorm. Dazu z�hlen auch die storybasierten Spiele, bei denen um die Handlungen des Spielers eine Geschichte aufgebaut wird, in die der Spieler eintaucht. Dabei ist es allerdings h�ufig so, dass der Spieler keine gro�e Wahl hat, was den Verlauf der Story angeht. Er muss zum Beispiel Gegenst�nde aufsammeln, Gegner besiegen und bestimmten Wegen folgen, um am Ende ans Ziel zu gelangen. Dieser vorbestimmte Verlauf f�hrt dazu, dass die Entscheidungsfreiheit des Spielers eher begrenzt ist. Denn l�sst er bestimmte Handlungen aus, geht es in der Geschichte nicht weiter.
\\
Mit dieser Arbeit soll ein Prototyp entstehen, der dem Spieler die angesprochene begrenzte Entscheidungsfreiheit nimmt und einen freien Willen suggeriert. Dabei soll untersucht werden, ob Manipulationstechniken, die in 3D erfolgreich angewendet werden, auch auf 2,5D �bertragbar sind. Der Spieler soll das Gef�hl haben, alle seine Entscheidungen eigenst�ndig getroffen zu haben und sich in jeder seiner Entscheidungen frei f�hlen. Tats�chlich aber wird vorab durch ein Analyselevel erfasst, welche Vorz�ge der Spieler hat, um so den Ausgang bestimmter Situationen vorhersagen zu k�nnen. Seine vermeintliche Handlungsfreiheit soll dann durch besagte Manipulationsmethoden erreicht werden. Teilweise stammen diese aus dem Bereich der Psychologie, teilweise aus bereits vorhandenen Computerspielen, die diese Thematik auch aufgreifen.
\\
Letztendlich soll die Frage beantwortet werden, ob es m�glich ist, den Spieler auch in 2,5D so zu manipulieren, dass er seine getroffenen Entscheidungen f�r seine eigenen h�lt. Dazu findet abschlie�end ein Interview mit den Probanden statt, bei dem ausgewertet wird, inwieweit die verwendeten Manipulationen erfolgreich waren bzw. was die Probanden veranlasst hat, sich in bestimmten Situationen f�r etwas zu entscheiden.

%
% Nachfolgende Sektionen nicht im Inhaltsverzeichnis anzeigen lassen!
\hideInContents
%
%................................................................
% Hier die Sektionen f�r die Einleitung

%................................................................
%
% Die kommenden Kapitel, Sektionen und Untersektionen wieder im Inhaltsverzeichnis anzeigen
\showInContents