%#####################################################################################################################
% Datei	: Einleitung.tex
% Autor	: Ulrike Uderhardt
%#####################################################################################################################
\chapter{Einleitung}
\label{chap:einleitung}

Schon vor Jahrzehnten haben Computerspiele in die Kinder- und Wohnzimmer unserer Welt Einzug genommen und sind heutzutage kaum noch wegzudenken. Dabei dienen Computerspiele haupts�chlich der Unterhaltung der Spieler. Die Vielfalt an unterschiedlichen Spiel-Genre ist enorm. Dazu z�hlen auch die storybasierten Spiele, bei denen um die Handlungen des Spielers eine Geschichte aufgebaut wird, in die der Spieler eintaucht. Dabei erf�hrt der Spieler meist sehr deutlich die Einschr�nkung, den Verlauf der Geschichte nicht beeinflussen zu k�nnen. Allerdings ist diese Begrenzung der Entscheidungsfreiheit unvermeidbar, da der Entwickler ein bestimmtes Geschehen verfolgen will. Der Spieler muss zum Beispiel Gegenst�nde aufsammeln, Gegner besiegen und bestimmten Wegen folgen, um am Ende ans Ziel zu gelangen. Ohne diese bestimmten Handlungen geht es in der Geschichte nicht weiter. Mit dieser Arbeit soll ein Prototyp eines 2,5D-Jump'n'Run-Spiels entstehen, der dem Spieler die angesprochene begrenzte Entscheidungsfreiheit suggeriert. Um dem Spieler solch ein Gef�hl der freien Wahl zu geben, soll eine Analyse seines Verhaltens erfolgen und darauf basierende Manipulationen den festgelegten Verlauf verschleiern. Teilweise stammen diese aus dem Bereich der Psychologie, teilweise aus bereits vorhandenen Computerspielen, die diese Thematik auch aufgreifen. Bisher finden Manipulationen jedoch gr��tenteils in 3D-Spielen eine breite Anwendung. In 2,5D-Spielen und speziell in Jump'n'Runs wurde dieser Bereich erst wenig behandelt. Auch sind wissenschaftliche Untersuchungen zum Thema Manipulationen im Allgemeinen bisher eher rar, insbesondere in Bezug auf dieses Spiel-Genre.
\\

Mithilfe von drei Erfolgskriterien soll der Prototyp bewertet werden. Dazu ist ein Fragebogen auszuarbeiten, den die Probanden nach dem Spielen ausf�llen sollen. Anhand einer Skala von \grqq trifft voll und ganz zu\grqq \ bis \grqq trifft �berhaupt nicht zu\grqq \ k�nnen die Probanden unterschiedliche Kriterien bewerten. Als Ziel sollen zun�chst 70\% der Spieler den Spa�faktor mit mindestens \grqq gut\grqq \ beurteilen. \grqq Gut\grqq \ entspricht im Fragebogen dem Feld \grqq trifft zu\grqq. Als n�chstes sollen sich die Spieler in mindestens 80\% der F�lle f�r den gew�nschten Weg entscheiden, was wiederum die Auswahlm�glichkeit \grqq trifft zu\grqq \ oder besser bedeutet. Und 66\% der Spieler sollen einen alternativen Weg bzw. ein alternatives Ende f�r m�glich halten.
\\

Das Ziel dieser Arbeit ist somit die Entwicklung eines Prototypen, um die zuvor gew�hlten Manipulationen evaluieren zu k�nnen. Dadurch soll die Frage, inwiefern sich die in 3D-Spielen verwendeten Manipulationen auch auf 2,5D-Jump'n'Runs �bertragen lassen, beantwortet werden k�nnen. Um die Manipulationen unter realistischeren Bedingungen testen zu k�nnen, soll eine m�glichst geschlossene Geschichte im Prototypen entwickelt werden.

%
% Nachfolgende Sektionen nicht im Inhaltsverzeichnis anzeigen lassen!
\hideInContents
%
%................................................................
% Hier die Sektionen f�r die Einleitung

%................................................................
%
% Die kommenden Kapitel, Sektionen und Untersektionen wieder im Inhaltsverzeichnis anzeigen
\showInContents