%#####################################################################################################################
% Datei	: Zusammenfassung.tex
% Autor	: Ulrike Uderhardt
%#####################################################################################################################

\chapter{Zusammenfassung und Ausblick}
\label{chap:zusammenfassung und ausblick}

An dieser Stelle soll noch einmal zusammenfassend ein �berblick �ber die Arbeit und deren Ergebnisse gegeben werden. Anschlie�end werden in einem Ausblick M�glichkeiten angesprochen, wie die Ergebnisse in zuk�nftigen Projekten noch verbessert werden k�nnen.

%................................................................
% Hier die Sektionen f�r die Zusammenfassung

\section{ Zusammenfassung}
\label{sec: zusammenfassung}


Ziel dieser Arbeit war es, zu �berpr�fen, ob die in 3D-Spielen verwendeten Manipulationen auch auf 2,5D-Jump-and-Runs �bertragbar sind. Dazu wurden zun�chst die Grundlagen von Manipulationen (siehe Kapitel \ref{chap:grundlagen manipulation}) erarbeitet, wobei festgestellt wurde, dass sowohl positive als auch (wenigere) negative Manipulationsstrategien verwendet werden sollten. Daraufhin konnten im Kapitel \ref{chap:verwandte arbeiten} geeignete Manipulationen aus bereits existierenden Spielen, sowie aus wenigen wissenschaftlichen Arbeiten und Artikeln erarbeitet werden, um sie im Prototypen umzusetzen. Dabei wurde sich daf�r entschieden, in einem ersten Level gewisse Eigenschaften und Spielweisen des Spielers zu erfassen, um das Spiel besser darauf aufbauen zu k�nnen. Die dadurch gemachten Feststellungen waren die Grundlagen f�r das Kapitel \ref{chap:konzeptionierung}. Die Konzeptionierung befasste sich dabei vor allem mit ausgew�hlten Manipulationsstrategien und den Spielereigenschaften. Darauf aufbauend wurde dann das Level-Design vorgestellt. \\
Im Kapitel \ref{chap:Planung} wurde noch einmal genau aufgef�hrt, wie und wodurch das Projekt realisiert wurde. Dazu z�hlen eine kurze Storybeschreibung, wie auch die vorgenommene Aufgabenverteilung, gesetzte Meilensteine und verwendete Werkzeuge.
Das Kapitel \ref{chap:Prototypische Umsetzung} befasste sich mit der Entwicklungsumgebung und Projektversionierung. Des weiteren wurden auch die implementierten Strukturen besprochen. \\
Mittels der Evaluierung (Kapitel \ref{chap:evaluierung}) konnte gezeigt werden, dass der Prototyp als erfolgreich bezeichnet werden kann. Dabei wurden sowohl die einzelnen Manipulationen, als auch die Erfolgskriterien insgesamt ausgewertet. Trotz der Beschr�nkungen des Prototypen waren die meisten Manipulationen gelungen und auch die festgelegten Erfolgskriterien in Bezug auf Spa� und Unauff�lligkeit der Manipulationen wurden erf�llt. Die Ergebnisse haben auch gezeigt, dass das Verh�ltnis zwischen positiven und negativen Manipulationen, wie in der Konzeption gefordert, angemessen war, da es kaum zu Frustrationen seitens der Probanden kam.


\section{ Ausblick}
\label{sec: ausblick}

Dennoch haben die Probandentests gezeigt, dass es noch einige Verbesserungsm�glichkeiten gibt, um die gew�nschte Wirkung einzelner Manipulationen optimieren zu k�nnen. Zun�chst ist es wichtig, dass der angezeigte Text in angemessener Schnelligkeit und Gr��e auf dem Bildschirm zu sehen ist, sodass auch wirklich jeder Spieler der Story folgen kann. In einer vollst�ndigen Implementierung sollte dazu auch auf eine realistischere Umsetzung, besonders hinsichtlich Grafik und Sound, gesetzt werden. Die Wegweiser wurden als eher verwirrend betrachtet und k�nnten durch eine Karte, auf der der Spieler sehen kann, wo er sich befindet, ausgetauscht werden. Gewisse Stellen in den einzelnen Leveln waren f�r die Spieler nicht gut genug zu erkennen. Dazu z�hlen vor allem der zerst�rbare Block und der Schein in der H�hle, die kaum von einem Spieler bemerkt wurden und somit f�r die Analyse der Manipulation nicht herangezogen werden konnten. Auch wichtig sind noch intensivere Tests der einzelnen Level, bevor die Probandenstudie beginnt. Denn nur so lassen sich Fehler wie der \grqq Vogelbug\grqq, der die sp�tere Bewertung der Manipulation einschr�nkte, verhindern. \\
Speziell das Analyselevel muss zudem �berarbeitet werden, um die Auswirkungen der angepassten Gespr�che und Level besser betrachten zu k�nnen. Einerseits k�nnen die Fehler durch falsch gesetzte Trigger behoben werden, andererseits m�ssen bestimmte Stellen im Level in Hinblick auf Sichtbarkeit und Erreichbarkeit noch einmal abgewandelt werden. Solche Analysen sollten aber auch immer beachten, dass die Selbsteinsch�tzung der Probanden auch sehr subjektiv ausfallen kann. Somit kann es auch mit der besten Umsetzung zu Schwankungen zwischen errechneter Analyse und gemachter Angabe des Spielers kommen. \\
In Zukunft sollte noch mehr auf l�ngerfristig wirkende Manipulationen gesetzt werden, da diese erfolgsversprechend erscheinen. Sie waren aber aufgrund der Beschr�nkungen des Prototypen kaum m�glich zu realisieren. Wichtig hierbei ist es, die Story noch weiter auszubauen und auch die Charaktere und Level komplexer zu gestalten, damit der Spieler eine st�rkere Bindung zum Spiel und deren Figuren bekommt. Erst so k�nnen gewisse Manipulationen, wie z.B. der Verrat von Antonio, richtig greifen.

%................................................................
